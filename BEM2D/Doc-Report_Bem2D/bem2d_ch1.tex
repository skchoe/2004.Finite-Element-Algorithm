\section{Formulation of Boundary Element Method on a Circle}
In this section, we define a differential equation which satisfies
Laplace equation defined on a circle $\gamma$ as follows:
\begin{itemize}
\item {\bf Domain} : A Disk of radius 1.
\item {\bf Boundary Condition} : {\bf Outer} normal vectors.
\item {\bf Analytic solution} : $u(x,y) = x^2 - y^2$.
\item {\bf Orientation of Integral} : Counter clock wise.
\end{itemize}

Note that, regardless of direction of normal vectors,
mathematically this formulation should gives an unique numerical
solution.

According to boundary element method, we can set up the following
relation:

\begin{eqnarray}
\frac{1}{2}u(x) = -\int_{\gamma} u(z) \nabla G(z,x)\cdot {\bf n}
dz + \int_{\gamma} \frac{\partial u}{\partial n}(z) G(z, x) dz.
\end{eqnarray}

By discretization with element $E_j$ on $\gamma$, $j=1 \dots N$,

\begin{eqnarray}
\frac{1}{2}u(x) =  &-&\sum_{j=1}^{N}\int_{E_j} u(z) \nabla
G(z,x)\cdot {\bf n} dz + \sum_{j=1}^{N}\int_{E_j} \frac{\partial
u}{\partial n}(z) G(z, x) dz.
\end{eqnarray}


Let ${x_1, \dots, x_N}$ be the center points of elements ${E_1,
\dots, E_N}$. The we can approximate above equation by

\begin{eqnarray}
\frac{1}{2}u(x_i) = - \sum_{j=1}^{N}u(x_j)\int_{E_j} \nabla
G(z,x_i)\cdot {\bf n} dz + \sum_{j=1}^{N}\frac{\partial
u}{\partial n}(x_j)\int_{E_j} G(z, x_i) dz.
\end{eqnarray}


Define
\begin{eqnarray}
\alpha_j^i &=& \int_{E_j}G(z,x_i)dz\\
\beta_j^i &=& \int_{E_j}\nabla G(z,x_i)\cdot {\bf n} dz.
\end{eqnarray}
Then we can rearrange above equation by putting unknowns to left
and known values to right side as follows:

\begin{eqnarray}
\frac{1}{2} u(x_i) &+& \sum_{j=1}^N \beta_j^iu(x_j) = \sum_{j=1}^N
\alpha^i_j\frac{\partial u}{\partial n}(x_j),
\end{eqnarray}
for all $i,j = 1 \dots N$.

This enables to set up the following system of linear equations
for $u(x_j), j = 0, \dot, N$.

\begin{eqnarray}
\label{mat1}
\begin{bmatrix}
    \frac{1}{2}+\beta^1_1& \beta^1_2& \cdots & \beta^1_N \\
    \beta^2_1& \frac{1}{2}+\beta^2_2& \cdots & \beta^2_N\\
    \vdots& \vdots& \ddots& \vdots\\
    \beta^N_1& \beta^N_2& \cdots & \frac{1}{2} + \beta^N_N \\
\end{bmatrix}
\begin{bmatrix}
    u(x_1)    \\
    u(x_2)    \\
    \vdots      \\
    u(x_N)    \\
\end{bmatrix}
=
\begin{bmatrix}
    \sum_{j=1}^N \alpha^1_j\frac{\partial u}{\partial n}(x_j)   \\
    \sum_{j=1}^N \alpha^2_j\frac{\partial u}{\partial n}(x_j)   \\
    \vdots  \\
    \sum_{j=1}^N \alpha^N_j\frac{\partial u}{\partial n}(x_j)   \\
\end{bmatrix}.
\end{eqnarray}

Note that $\{\alpha^i_j, \beta^i_j\}$ in  (\ref{mat1}) assume the
condition defined in the beginning(CCW, Outer normal).

Mathematically, the original problem has same result as the case
when we have both clockwise orientation and inward normal vector.
In boundary element formulation, this fact can be seen through the
change in (\ref{mat1}), since $\{\alpha^i_j, \frac{\partial
u}{\partial n}(x_j)\}$ have opposite signs but $\{\beta^i_j\}$ has
no sign change.

\begin{itemize}
\item Radius 1cm
\item Number of elements: $2^6 \cdots 2^{12}$
\item Slope: 1.45xx
\end{itemize}

\begin{figure}[h]
    \begin{center}
    \epsfig{file = circle_htest_6_12.eps, width = 12cm}
    \caption{This figure is showing the h-convergence test for above problem}
    \end{center}
\end{figure}


\section{Case: Clockwise Orientation with Outer normal}
In this section we define shortly a slightly different example
based on the initial definition of example at previous section.

\begin{itemize}
\item {\bf Domain} : A Disk of radius 1.
\item {\bf Boundary Condition} : {\bf Outer} normal vectors.
\item {\bf Analytic solution} : $u(x,y) = x^2 - y^2$.
\item {\bf Orientation of Integral} : Clock wise.
\end{itemize}

This cause the changes to each element of formulation above:
\begin{itemize}
\item {\bf $\alpha^i_j$} : Sign Change
\item {\bf $\beta^i_j$} : Sign Change
\item {\bf $\frac{\partial u}{\partial n}(x_j)$} : No Change.
\end{itemize}

With the same meaning of notation as (\ref{mat1}), we can set up
the system of linear equation for this problem by modifying signs:


\begin{eqnarray}
\label{mat2}
\begin{bmatrix}
    \frac{1}{2}-\beta^1_1& -\beta^1_2& \cdots & -\beta^1_N \\
    -\beta^2_1& \frac{1}{2}-\beta^2_2& \cdots & -\beta^2_N\\
    \vdots& \vdots& \ddots& \vdots\\
    -\beta^N_1& -\beta^N_2& \cdots & \frac{1}{2} - \beta^N_N \\
\end{bmatrix}
\begin{bmatrix}
    u(x_1)    \\
    u(x_2)    \\
    \vdots      \\
    u(x_N)    \\
\end{bmatrix}
=
\begin{bmatrix}
    - \sum_{j=1}^N \alpha^1_j\frac{\partial u}{\partial n}(x_j)   \\
    - \sum_{j=1}^N \alpha^2_j\frac{\partial u}{\partial n}(x_j)   \\
    \vdots  \\
    - \sum_{j=1}^N \alpha^N_j\frac{\partial u}{\partial n}(x_j)   \\
\end{bmatrix}.
\end{eqnarray}

\section{Question}
{\bf Above two cases (\ref{mat1} and \ref{mat2}) has only
difference in its orientation. But two solutions from \ref{mat1}
and \ref{mat2} are definitely different. Is there anything wrong
on the formulation of second problem(clockwise orientation)? I
wonder if Gauss theorem presume a type of orientation or direction
of normal vectors.}

\section{Computation of $\alpha_j(x_i), \beta_j(x_i)$}
We set
\begin{eqnarray}
(x_1, x_2) &=& R (\cos(\theta_1), \sin(\theta_1))\\
(z_1, z_2) &=& r (\cos(\theta_2), \sin(\theta_2))
\end{eqnarray}

\subsection{Nonsingular $\alpha$}
\begin{eqnarray}
\alpha_j(x) &=& \int_{E_j} - \frac{1}{2\pi} \log | x-z | dx \\
&=&- \frac{1}{2\pi} \int_{E_j} \frac{1}{2} \log\left[(x_1-z_1)^2 + (x_2-z_2)^2 \right] dx \\
&=&- \frac{1}{2\pi} r |J_j| \frac{1}{2} \left[ \sum_{k} \log \{(R \cos(\theta_1)-r\cos(\theta_2(\xi_k))^2 + (R\sin(\theta_1)-r\sin(\theta_2(\xi_k))^2 \} \cdot w_k \right].
\end{eqnarray}

\subsection{Singular $\alpha$}
In this case we have $R = r$. This enables:
\begin{equation}
|x-z| = 2 R \sin(\frac{|\theta_1 - \theta_2|}{2}).
\end{equation}

\begin{eqnarray}
\alpha_j(x_i) &=& - \frac{R}{2\pi} \int_{\theta_j}^{\theta_{j+1}}  \log \left[2R\sin(\frac{|\theta - \theta_i|}{2}) \right] d\theta \\
&=&- \frac{R}{2\pi} \int_{\theta_j}^{\theta_{j+1}}  \log\left[\frac{\sin(\frac{|\theta - \theta_i|}{2})}{\frac{|\theta - \theta_i|}{2}}\right] + \log(R |\theta-\theta_i|)d\theta \\
&=&- \frac{R}{2\pi} |J_j| \left[\sum_k\log(\frac{\sin(\frac{|\theta -\theta_i|}{2})}{\frac{|\theta -\theta_i|}{2}})\right]-\frac{R}{2\pi}\left[\int_{\theta_j}^{\theta_{j+1}}\log(R|\theta-\theta_i|) d\theta \right].
\end{eqnarray}
Note that
\begin{eqnarray}
&&-\frac{R}{2\pi}\left[\int_{\theta_j}^{\theta_{j+1}}\log(R|\theta-\theta_i|) d\theta \right]\\
&=&-\frac{R}{2\pi}\left[\int_{\theta_j}^{\theta_i}\log\{R(\theta_i-\theta)\}d\theta+\int_{\theta_i}^{\theta_{j+1}}\log\{R(\theta-\theta_i)\}d\theta\right]\\
&=&-\frac{R}{2\pi}\left[\int_{R(\theta_i-\theta_j}^{0}\log\psi(-\frac{1}{R})d\psi+\int_{0}^{R(\theta_{j+1}-\theta_i}\log\psi \frac{1}{R} d\psi \right]\\
&=&-\frac{R}{2\pi} \frac{1}{R} \left[\int_{0}^{R(\theta_i-\theta_j)}\log\psi d\psi+\int_{0}^{R(\theta_{j+1}-\theta_i)}\log\psi d\psi \right]\\
&=&-\frac{1}{2\pi} \left[ (\theta_i-\theta_j)\{ \log\{R(\theta_i-\theta_j)\} - 1 \} + (\theta_{j+1}-\theta_i) \{ \log\{R(\theta_{j+1}-\theta_i)\}-1\}\right]\\
\end{eqnarray}

\subsection{Nonsingular $\beta$}
\begin{eqnarray}
G(z, x) &=& -\frac{1}{2\pi}\log|z-x| = -\frac{1}{2\pi} \frac{1}{2} \log\left[ (z_1-x_1)^2 + (z_2-x_2)^2 \right] \\
G_{z_1} &=& -\frac{1}{2\pi}\frac{z_1-x_1}{(z_1-x_1)^2 +(z_2-x_2)^2}\\
G_{z_2} &=& -\frac{1}{2\pi}\frac{z_2-x_2}{(z_1-x_1)^2 +(z_2-x_2)^2}\\
\end{eqnarray}

\begin{eqnarray}
\nabla_z G(z, x) \cdot {\bf n} &=& G_{z_1} \cos\theta_z + G_{z_2} \sin\theta_z\\
&=&-\frac{1}{2\pi}\frac{(r\cos\theta_z-R\cos\theta_x)\cos\theta_z+(r\sin\theta_z-R\sin\theta_x)\sin\theta_z}{(r\cos\theta_z-R\cos\theta_x)^2+(r\sin\theta_z- R\sin\theta_x)^2}.
\end{eqnarray}

\begin{eqnarray}
&&\beta_j(x_i) = R \int_{\theta_j}^{\theta_{j+1}} \nabla G(z,x)\cdot {\bf n} d\theta \\
&=& R |J_j| (-\frac{1}{2\pi}) \sum_k\frac{(r\cos\theta_z(\xi_k)-R\cos\theta_x)\cos\theta_z(\xi_k)+(r\sin\theta_z(\xi_k)-R\sin\theta_x)\sin\theta_z(\xi_k)}{(r\cos\theta_z(\xi_k)-R\cos\theta_x)^2+(r\sin\theta_z(\xi_k)-R\sin\theta_x)^2} \cdot w_k.
\end{eqnarray}

\subsection{Singular $\beta$}
\begin{eqnarray}
\beta_j(x_i) &=& R \int_{\theta_j}^{\theta_{j+1}} -\frac{1}{2\pi}\frac{R(\cos\theta_z-\cos\theta_x)\cos\theta_z+R(\sin\theta_z-\sin\theta_x)\sin\theta_z}{R^2(\cos\theta_z-\cos\theta_x)^2+R^2(\sin\theta_z-\sin\theta_x)^2}d\theta\\
&=&-\frac{1}{2\pi} \frac{1}{2} (\theta_{j+1} - \theta_j)
\end{eqnarray}

\section{Check list for debugging Bem with constant basis on a
circle}

\begin{itemize}
\item alpha - nonsing
    \begin{enumerate}
    \item direct testing - Test passed
    \end{enumerate}
\item alpha - sing
    \begin{enumerate}
    \item direct testing - On testing
    \end{enumerate}
        \begin{enumerate}
        \item 1) trepezoidal / simson's               a) b)
        \item 2) trepezoidal / my quadrature  a) b)
        \item 3) simson's    / my quadrature  a) b)
        \end{enumerate}

\item beta - nonsing
    \begin{itemize}
    \item direct testing - Test passed, yet the case with radC=radT untested.
    \item circuit testing - needless
    \end{itemize}

\item beta - sing
    \begin{itemize}
    \item direct testing - Test passed
    \item circuit testing - Test
    \end{itemize}
\end{itemize}
