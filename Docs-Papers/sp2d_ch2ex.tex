

\{ {\it  sp2d\_ch2ex.tex} \}

\subsection{Example of the Method}

In this section we look into the numerical method in practice.
Let's think of the case:
\begin{equation}
N_r = 1, \;\;\;\; P = 2 \;\;\;\; N_\theta = 4,
\end{equation}
that is we have only one element in $R$-direction with order 2 and 4 modes in $\theta$-direction. In this case, we the local and global basis function for spectral polynomial method is identical.

The differential equation
\begin{eqnarray}
- \left(\frac{\partial^2}{\partial r^2} + \frac{1}{r} \frac{\partial}{\partial r} + \frac{1}{r^2}\frac{\partial^2}{\partial \theta^2} \right) u(r,\theta) = f(r, \theta)  \\
\end{eqnarray}
is defined for  $(r, \theta) \in [a, b] \times (0, 2\pi]$ with the boundary condition:

\begin{eqnarray}
du(a,\theta) &=& g_1(\theta) \\
\frac{\partial}{\partial r} u(b,\theta) &=& g_2(\theta)
\end{eqnarray}

We approximate the numerical solution of this using the 3 1st
order basis functions:

\begin{eqnarray}
\psi_j(x) = \left \{
    \begin{array}{ll}
    \frac{1-\chi^{-1}(x)}{2} & j=0, \\
    (\frac{1-\chi^{-1}(x)}{2})(\frac{1+\chi^{-1}(x)}{2})P_0^{1,1}(\chi^{-1}(x)) & j=1 \\
    \frac{1+\chi^{-1}(x)}{2} & j=2. \\
    \end{array} \right.
 \end{eqnarray}

where $\chi$ is a parametric linear mapping from $[-1, 1]$, to $[a, b]$.

Since $P=2$, the total degree of freedom for each angle $\theta$ is $3$.
We define the solution $u(r,\theta)$ to be the form:
\begin{equation}
u(r, \theta) = \sum_{j=0}^2 \sum_{k=-1}^2 \hat u_{j,k} \psi_j(r) e^{ik\theta}.
\end{equation}

%By the Trefthen's book, we assume
%\begin{equation}
%u(r, \theta) = \sum_{j=0}^4 \sum_{k=-2}^2 \tilde u_{j,k}^g \psi_j^g(r) e^{ik\theta}.
%\end{equation}
%where
%\begin{eqnarray}
%\tilde u_{j,k}^g = \left \{
%    \begin{array}{ll}
%    \hat u_{j,N/2}^g/2,  & k=-N/2, \\
%    \hat u_{j,k}^g,      & -N/2+1 \le k \le N/2-1\\
%   \hat u_{j,N/2}^g/2,  & k=N/2, \\
%    \end{array} \right.
% \end{eqnarray}
To apply Dirichlet condition note that
\begin{eqnarray}
\psi_j(a) = \left \{
    \begin{array}{ll}
    1 & j = 0, \\
    0 & j = 1, 2 \\
    \end{array} \right.
\end{eqnarray}

So we have
\begin{equation}
u(a, \theta) = \sum_{k=-1}^2 \hat u_{0,k} \psi_0(a) e^{ik\theta}  = \sum_{k=-1}^2 \hat u_{0,k} e^{ik\theta}.
\end{equation}

We finally obtain these relationships:

\begin{eqnarray}
\begin{bmatrix}
    g_1(\frac{\pi}{2}) \\
    g_1(\pi) \\
    g_1(\frac{3\pi}{2}) \\
    g_1(2\pi) \\
\end{bmatrix}
=
\begin{bmatrix}
    e^{i(-1)\frac{\pi}{2}} & e^{i(0)\frac{\pi}{2}} & e^{i(1)\frac{\pi}{2}} & e^{i(2)\frac{\pi}{2}} \\
    e^{i(-1)\pi}           & e^{i(0)\pi}           & e^{i(1)\pi}           & e^{i(2)\pi} \\
    e^{i(-1)\frac{3\pi}{2}} & e^{i(0)\frac{3\pi}{2}} & e^{i(1)\frac{3\pi}{2}} & e^{i(2)\frac{3\pi}{2}} \\
    e^{i(-1)2\pi}           & e^{i(0)2\pi}           & e^{i(1)2\pi}           & e^{i(2)2\pi} \\
\end{bmatrix}
\begin{bmatrix}
    \hat u_{0,-1} \\
    \hat u_{0,0} \\
    \hat u_{0,1} \\
    \hat u_{0,2} \\
\end{bmatrix}
=
\begin{bmatrix}
    -i&1&i&-1\\
    -1&1&-1&1\\
    i&1&-i&-1\\
    1&1&1&1\\
\end{bmatrix}
\begin{bmatrix}
    \hat u_{0,-1} \\
    \hat u_{0,0} \\
    \hat u_{0,1} \\
    \hat u_{0,2} \\
\end{bmatrix}
\end{eqnarray}

By direct inversion of the matrix we have 4 unknowns
\begin{eqnarray}
\begin{bmatrix}
    \hat u_{0,-1} \\
    \hat u_{0,0} \\
    \hat u_{0,1} \\
    \hat u_{0,2} \\
\end{bmatrix}
= \frac{1}{4}
\begin{bmatrix}
    i&-1&-i&1\\
    1&1&1&1\\
    -i&-1&i&1\\
    -1&1&-1&1\\
\end{bmatrix}
\begin{bmatrix}
    g_1(\frac{\pi}{2}) \\
    g_1(\pi) \\
    g_1(\frac{3\pi}{2}) \\
    g_1(2\pi) \\
\end{bmatrix}
\end{eqnarray}

For Newmann Boundary condition, from equation \ref{bdyt}, we had

\begin{eqnarray} \label{bdyt}
&\sum_j \sum_k \hat{u}_{jk} \delta_{qk} \left( ({\bf M}_3)_{pj} + ({\bf M}_2)_{pj} + k^2 ({\bf M}_1)_{pj} \right)\\
= &T_4 + \int_0^{2\pi} \sum_j \sum_k \hat{u}_{jk} e^{iq\theta} e^{ik\theta} \left[r^2 \phi_p(r) \frac{d}{dr}\phi_j(r)\right]_a^b d\theta
%= &T_4 + \sum_j \sum_k \hat{u}_{jk} \delta_{qk} \left[b^2\phi_p(b)\frac{d}{dr}\phi_j(b) - a^2\phi_p(a)\frac{d}{dr}\phi_j(a) \right]
\end{eqnarray}

In this equation, we can drive the boundary term as follows:
\begin{eqnarray} \label{bdyt}
&\int_0^{2\pi} \sum_j \sum_k \hat{u}_{jk} e^{iq\theta} e^{ik\theta} \left[r^2 \phi_p(r) \frac{d}{dr}\phi_j(r)\right]_a^b d\theta \\
=&b^2\phi_p(b) \int_0^{2\pi} \sum_j \sum_k \hat{u}_{jk} \frac{d}{dr}\phi_j(b) e^{ik\theta} e^{iq\theta} d\theta \\
-&a^2\phi_p(a) \int_0^{2\pi} \sum_j \sum_k \hat{u}_{jk} \frac{d}{dr}\phi_j(a) e^{ik\theta} e^{iq\theta} d\theta \\
=&b^2\phi_p(b) \int_0^{2\pi} \frac{d}{dr}u(b, \theta) e^{iq\theta} d\theta \\
-&a^2\phi_p(a) \int_0^{2\pi} \frac{d}{dr}u(a, \theta) e^{iq\theta} d\theta \\
=&b^2\phi_p(b) \int_0^{2\pi} \frac{d}{dr}u(b, \theta) e^{iq\theta} d\theta \\
-&a^2\phi_p(a) \int_0^{2\pi} \frac{d}{dr}u(a, \theta) e^{iq\theta} d\theta \\
=&b^2\phi_p(b) \widehat{ \frac{d}{dr}u(b) }_l \\
-&a^2\phi_p(a) \widehat{ \frac{d}{dr}u(a) }_l \\
\end{eqnarray}

We now apply these two conditions:

For $k = -1, 0, 1, 2$, with the Dirichlet condition we have the matrices:

%\begin{gather}
\begin{eqnarray}
\mathcal{M}^{(k)} &=& \begin{bmatrix}
                        ({\bf M}_3)_{pj} + ({\bf M}_2)_{pj} + k^2 ({\bf M}_1)_{pj}
                  \end{bmatrix}_{p = 0, j = 0}^{2,2}, \\
\mathbf{\hat u}_k &=& \begin{bmatrix}
                        \hat u_{0,k} \\
                        \hat u_{1,k} \\
                        \hat u_{2,k} \\
                    \end{bmatrix},   \\
\mathcal{F}_k &=& \begin{bmatrix}
                        \iint f(r, \theta)\phi_0(r)e^{ik\theta}d\theta dr \\
                        \iint f(r, \theta)\phi_1(r)e^{ik\theta}d\theta dr \\
                        \iint f(r, \theta)\phi_2(r)e^{ik\theta}d\theta dr \\
                    \end{bmatrix}
               =  \begin{bmatrix}
                        \sum_\sigma w_\sigma r_\sigma^2\phi_0(r_\sigma) \widehat{f(r_\sigma)}_k \\
                        \sum_\sigma w_\sigma r_\sigma^2\phi_1(r_\sigma) \widehat{f(r_\sigma)}_k \\
                        \sum_\sigma w_\sigma r_\sigma^2\phi_2(r_\sigma) \widehat{f(r_\sigma)}_k \\
                    \end{bmatrix}\\
\mathcal{GB}_k &=& b^2  \widehat{u^\prime(b)}_l
                    \begin{bmatrix}
                        \phi_0(b)\\
                        \phi_1(b)\\
                        \phi_2(b)\\
                    \end{bmatrix} \\
\mathcal{GA}_k &=& a^2  \widehat{u^\prime(a)}_l
                    \begin{bmatrix}
                        \phi_0(a)\\
                        \phi_1(a)\\
                        \phi_2(a)\\
                    \end{bmatrix} \\
\end{eqnarray}
%\end{gather}

Recall that we have $\mathcal{F}_{-1}, \mathcal{F}_{0}, \mathcal{F}_{1}, \mathcal{F}_{2}$.

For fixed $\sigma$, define a vector

\begin{equation}
\widehat{f(r_\sigma,\theta)}= \begin{bmatrix}
                                \widehat{f(r_\sigma,\theta_1)} \\
                                \widehat{f(r_\sigma,\theta_2)} \\
                                \widehat{f(r_\sigma,\theta_3)} \\
                                \widehat{f(r_\sigma,\theta_4)} \\
                              \end{bmatrix}
\end{equation}
where $\theta_k = k * \frac{\pi}{2}, k=1,2,3,4$.

Since we know $\hat u_{0, k}$ for each $k$, we can modify above
elements so that the relation become an augmented system.

Then we obtain the following global system.

\begin{eqnarray}
\begin{bmatrix}
    \mathcal{M}^{(-1)}& 0             & 0             & 0 \\
    0              & \mathcal{M}^{(0)}& 0             & 0 \\
    0              & 0              & \mathcal{M}^{(1}) & 0 \\
    0              & 0              & 0             & \mathcal{M}^{(2)}.  \\
\end{bmatrix}
\begin{bmatrix}
    \mathbf{\hat u}_{-1}\\
    \mathbf{\hat u}_{0}\\
    \mathbf{\hat u}_{1}\\
    \mathbf{\hat u}_{2}\\
\end{bmatrix}
=
\begin{bmatrix}
    \mathcal{F}_{-1}\\
    \mathcal{F}_{0}\\
    \mathcal{F}_{1}\\
    \mathcal{F}_{2}\\
\end{bmatrix}
+
\begin{bmatrix}
    \mathcal{GB}_{-1}\\
    \mathcal{GB}_{0}\\
    \mathcal{GB}_{1}\\
    \mathcal{GB}_{2}\\
\end{bmatrix}
-
\begin{bmatrix}
    \mathcal{GA}_{-1}\\
    \mathcal{GA}_{0}\\
    \mathcal{GA}_{1}\\
    \mathcal{GA}_{2}\\
\end{bmatrix}
\end{eqnarray}

From this, we obtain $\left[ \mathbf{\hat u}_{-1}, \mathbf{\hat u}_{0},\mathbf{\hat u}_{1},\mathbf{\hat u}_{2} \right]$.

Since we originally set the solution to be of the form:
\begin{equation}
u(r_\sigma, \theta_\tau) = \sum_{j = 0}^{4}\sum_{k = -2}^{2} \hat u_{jk}\psi_j(r_\sigma)e^{ik\theta_\tau},
\end{equation}
this have the matrix form as follows:
\begin{eqnarray}
\begin{bmatrix}
    u(r_1, \theta_1)&u(r_2, \theta_1)&u(r_3, \theta_1) \\
    u(r_1, \theta_2)&u(r_2, \theta_2)&u(r_3, \theta_2) \\
    u(r_1, \theta_3)&u(r_2, \theta_3)&u(r_3, \theta_3) \\
    u(r_1, \theta_4)&u(r_2, \theta_4)&u(r_3, \theta_4) \\
\end{bmatrix} \\
=
\begin{bmatrix}
    ifft(u_0)_1 & ifft(u_1)_1 & ifft(u_2)_1 & ifft(u_3)_1 & ifft(u_4)_1 & \\
    ifft(u_0)_2 & ifft(u_1)_2 & ifft(u_2)_2 & ifft(u_3)_2 & ifft(u_4)_2 & \\
    ifft(u_0)_3 & ifft(u_1)_3 & ifft(u_2)_3 & ifft(u_3)_3 & ifft(u_4)_3 & \\
    ifft(u_0)_4 & ifft(u_1)_4 & ifft(u_2)_4 & ifft(u_3)_4 & ifft(u_4)_4 & \\
\end{bmatrix}
\begin{bmatrix}
    \psi_0(r_1) & \psi_0(r_2) & \psi_0(r_3) \\
    \psi_1(r_1) & \psi_1(r_2) & \psi_1(r_3) \\
    \psi_2(r_1) & \psi_2(r_2) & \psi_2(r_3) \\
    \psi_3(r_1) & \psi_3(r_2) & \psi_3(r_3) \\
    \psi_4(r_1) & \psi_4(r_2) & \psi_4(r_3) \\
\end{bmatrix}
\end{eqnarray}
