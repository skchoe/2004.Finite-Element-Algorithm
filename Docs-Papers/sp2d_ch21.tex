

\{ {\it  sp2d\_ch21.tex} \}

\subsection{2 Dimensional Annulus Problem}

\begin{eqnarray}
- \left(\frac{\partial^2}{\partial r^2} + \frac{1}{r} \frac{\partial}{\partial r} + \frac{1}{r^2}\frac{\partial^2}{\partial \theta^2} \right) u(r,\theta) = f(r,\theta) \\
u(r=a,\theta) = g_1(\theta), \;\; \theta \in [0,2\pi] \\
\frac{\partial}{\partial r} u(r=b,\theta) = 0, \;\; \theta \in [0,2\pi] \\
u(r,\theta=0) = u(r,\theta=2\pi), \;\; r\in[a,b] \\
\int_0^{2\pi} u(r,\theta) d\theta = \mathbf{\int_o^{2\pi} g_1(\theta) d\theta}, \;\; r\in[a,b]
\end{eqnarray}



Assume that $N$ is an even number, $x_j = \frac{2\pi}{N}j, \; j=1,\ldots,N$, then:

\begin{eqnarray}
\hat{u}_k = \frac{1}{2\pi} \int_0^{2\pi}  e^{-ikx} u(x) dx  \simeq \frac{1}{2\pi}\frac{2\pi}{N} \sum_{j=0}^{N-1} e^{-ikx_j} u(x_j),  \;\; k=-\frac{N}{2}+1,\ldots,\frac{N}{2}\\
u_j = u(x_\mathbf{j}) = \sum_{k=-N/2+1}^{N/2} e^{ikx_j}\hat{u}_k, \;\; j=1,\ldots,N.
\end{eqnarray}

We have the orthogonal basis $\{e^{ik\theta}\}_{k = 0}^{N-1}$ with the following orthogonality:
\begin{equation}
\frac{1}{2\pi} \int_{0}^{2\pi} e^{ik\theta} e^{iq\theta} d\theta = \delta_{kq}.
\end{equation}



The representation of $u$ is guaranteed by Weierstrass theorem:
\begin{equation}
u(r,\theta) = \sum_{j=0}^{N_r} \sum_{k=-N_\theta/2+1}^{N_\theta/2} \hat{u}_{jk} \phi_j(r) e^{ik\theta}
\end{equation}

Then the Poisson equation and its modified form of polar coordinate was obtained:
\begin{eqnarray}
- \frac{\partial^2}{\partial r^2}u(r,\theta) - \frac{1}{r} \frac{\partial}{\partial r} u(r,\theta) - \frac{1}{r^2}\frac{\partial^2}{\partial \theta^2} u(r,\theta) &=& f(r,\theta) \\
- r^2 \frac{\partial^2}{\partial r^2} u(r,\theta) - r \frac{\partial}{\partial r} u(r,\theta) - \frac{\partial^2}{\partial \theta^2} u(r,\theta) &=& r^2 f(r,\theta)
\end{eqnarray}


Let us define the following matrices:
\begin{eqnarray}
({\bf M}_1)_{ij} & = & \int_a^b \phi_i(r) \phi_j(r) \;dr  \\
({\bf M}_2)_{ij} & = & \int_a^b r \phi_i(r) \frac{d}{dr} \phi_j(r) \;dr  \\
({\bf M}_3)_{ij} & = & \int_a^b r^2  \frac{d}{dr} \phi_i(r)  \frac{d}{dr} \phi_j(r) \;dr
\end{eqnarray}

To apply Galerkin method, test functions are of the form:
\begin{equation}
\phi_p(r) e^{iq\theta}, \;\; p=0,\ldots,N_r, \;\; q = -\frac{N_\theta}{2}+1,\ldots,\frac{N_\theta}{2}
\end{equation}

We have the weak form of the equation:
\begin{equation}
-\langle  r^2 \frac{\partial^2}{\partial r^2} u(r,\theta) + r \frac{\partial}{\partial r} u(r,\theta) + \frac{\partial^2}{\partial \theta^2} u(r,\theta), \phi_p(r) e^{iq\theta} \rangle = \langle r^2 f(r,\theta), \phi_p(r) e^{iq\theta} \rangle.
\end{equation}

\begin{eqnarray}
&-& \int_a^b \int_0^{2\pi} \phi_p(r) e^{iq\theta} r^2 \frac{\partial^2}{\partial r^2} u(r,\theta) d\theta dr \\
&-& \int_a^b \int_0^{2\pi} \phi_p(r) e^{iq\theta} r \frac{\partial}{\partial r} u(r,\theta) d\theta dr \\
&-& \int_a^b \int_0^{2\pi} \phi_p(r) e^{iq\theta} \frac{\partial^2}{\partial \theta^2} u(r,\theta) d\theta dr  \\
&=& \int_a^b \int_0^{2\pi} \phi_p(r) e^{iq\theta} r^2 f(r,\theta) d\theta dr \\
\end{eqnarray}


\begin{eqnarray}
T_1 &=& \int_a^b \int_0^{2\pi} \phi_p(r) e^{iq\theta} r^2 \frac{\partial^2}{\partial r^2} u(r,\theta) d\theta dr \\
    &=& \int_0^{2\pi} \sum_j \sum_k \hat{u}_{jk} e^{iq\theta} e^{ik\theta} \int_a^b r^2 \frac{d^2}{dr^2} \phi_j(r) \phi_p(r) dr d\theta\\
    &=& \int_0^{2\pi} \sum_j \sum_k \hat{u}_{jk} e^{iq\theta} e^{ik\theta} \left( \left[r^2 \phi_p(r) \frac{d}{dr}\phi_j(r)\right]_a^b -
        \int_a^b \left(2 r \phi_p(r) \frac{d}{dr}\phi_j(r) + r^2 \frac{d}{dr}\phi_p(r) \frac{d}{dr}\phi_j(r)\right) dr \right) d\theta \\
    &=& \int_0^{2\pi} \sum_j \sum_k \hat{u}_{jk} e^{iq\theta} e^{ik\theta} \left( \left[r^2 \phi_p(r) \frac{d}{dr}\phi_j(r)\right]_a^b - 2 ({\bf M}_2)_{pj} - ({\bf M}_3)_{pj} \right) d\theta \\
    &=& 2\pi \sum_j \sum_k \hat{u}_{jk} \delta_{qk} \left( \left[r^2 \phi_p(r) \frac{d}{dr}\phi_j(r)\right]_a^b - 2 ({\bf M}_2)_{pj} - ({\bf M}_3)_{pj} \right)
\end{eqnarray}

\begin{eqnarray}
T_2 &=& \int_a^b \int_0^{2\pi} \phi_p(r) e^{iq\theta} r \frac{\partial}{\partial r} u(r,\theta) d\theta dr \\
    &=& \int_0^{2\pi} \sum_j \sum_k \hat{u}_{jk} e^{iq\theta} e^{ik\theta} \int_a^b r \frac{d}{dr} \phi_j(r) \phi_p(r) dr d\theta\\
    &=& \int_0^{2\pi} \sum_j \sum_k \hat{u}_{jk} e^{iq\theta} e^{ik\theta} ({\bf M}_2)_{pj} d\theta\\
%    &=& \int_0^{2\pi} \sum_j \sum_k \hat{u}_{jk} e^{iq\theta} e^{ik\theta} \left( \left[r \phi_p(r) \phi_j(r)\right]_a^b -
%       \int_a^b \phi_j(r) \frac{d}{dr}\left( r \phi_p(r)\right) dr \right) d\theta \\
%    &=& \int_0^{2\pi} \sum_j \sum_k \hat{u}_{jk} e^{iq\theta} e^{ik\theta} \left( \left[r \phi_p(r) \phi_j(r)\right]_a^b -
%       \int_a^b \left( \phi_j(r) \phi_p(r) + r \phi_j(r) \frac{d}{dr}\phi_p(r) \right) dr \right) d\theta \\
%    &=& \int_0^{2\pi} \sum_j \sum_k \hat{u}_{jk} e^{iq\theta} e^{ik\theta} \left( \left[r \phi_p(r) \phi_j(r)\right]_a^b - ({\bf M}_1)_{pj} - ({\bf M}_2)_{jp} \right) d\theta \\
%    &=& 2\pi \sum_j \sum_k \hat{u}_{jk} \delta_{qk} \left( \left[r \phi_p(r) \phi_j(r)\right]_a^b - ({\bf M}_1)_{pj} - ({\bf M}_2)_{jp} \right)
\end{eqnarray}

\begin{eqnarray}
T_3 &=& \int_a^b \int_0^{2\pi} \phi_p(r) e^{iq\theta} \frac{\partial^2}{{\partial \theta}^2} u(r,\theta) d\theta dr \\
    &=& \int_0^{2\pi} \sum_j \sum_k \hat{u}_{jk} e^{iq\theta} \frac{\partial^2}{{\partial \theta}^2} e^{ik\theta} \int_a^b \phi_j(r) \phi_p(r) dr d\theta\\
    &=& \int_0^{2\pi} \sum_j \sum_k \hat{u}_{jk} e^{iq\theta} (-k^2) e^{ik\theta} ({\bf M_1})_{pj} d\theta\\
    &=& 2\pi \sum_j \sum_k \hat{u}_{jk} (-k^2) \delta_{qk} ({\bf M_1})_{pj}
\end{eqnarray}

\begin{eqnarray}
T_1 &=& 2\pi \sum_j \sum_k \hat{u}_{jk} \delta_{qk} \left(- 2 ({\bf M}_2)_{pj} - ({\bf M}_3)_{pj} \right) \\
    &+& 2\pi \sum_j \sum_k \hat{u}_{jk} \delta_{qk} \left[ b^2\phi_p(b)\frac{d}{dr}\phi_j(b) - a^2\phi_p(a)\frac{d}{dr}\phi_j(a) \right] \\
T_2 &=& 2\pi \sum_j \sum_k \hat{u}_{jk} \delta_{qk} ({\bf M}_2)_{pj} \\
%T_2 &=& 2\pi \sum_j \sum_k \hat{u}_{jk} \delta_{qk} \left( - ({\bf M}_1)_{pj} - ({\bf M}_2)_{jp} \right) \\
%    &+& 2\pi \sum_j \sum_k \hat{u}_{jk} \delta_{qk} \left[ b \phi_p(b) \phi_j(b) - a \phi_p(a) \phi_j(a) \right] \\
T_3 &=& 2\pi \sum_j \sum_k \hat{u}_{jk} \delta_{qk} (-k^2) ({\bf M}_1)_{pj}
\end{eqnarray}



Now the form $T_4$ is as follows.

\begin{eqnarray}
T_4 &=& \int_a^b \int_0^{2\pi} \phi_p(r) e^{iq\theta} r^2 f(r,\theta) d\theta dr \\
    &=& \int_a^b r^2\phi_p(r) \int_0^{2\pi} f(r,\theta) e^{iq\theta} d\theta dr \\
\end{eqnarray}

For given $r$, the Discrete Fourier Transform for $f(r, \theta)$ is defined by

\begin{equation}
f(r, \theta_\tau) = \frac{1}{2\pi} \sum_{k=-N_\theta/2+1}^{N_\theta/2} e^{ik\theta_\tau} \widehat{f(r)}_k
\end{equation}

where
\begin{equation}
\widehat{f(r)}_k = \frac{2\pi}{N_\theta} \sum_{j=1}^{N_\theta} e^{-ik\theta_j} f(r, \theta_j)
\end{equation}
with $ k \in \{-\frac{N_\theta}{2}+1, \cdots, \frac{N_\theta}{2}\}$ and $\theta_j \in \{ \frac{2\pi}{N_\theta}, \cdots, 2\pi  \}$.

As boundary conditions, we have the following:
\begin{eqnarray}
du(a,\theta) &=& g_1(\theta) \\
\frac{\partial}{\partial r} u(b,\theta) &=& g_2(\theta)
\end{eqnarray}

%To obtain zero derivative at each sawtooth point, we define $\widehat{f(r)}_k, k = -N_\theta/2, N_\theta/2$  as

%\begin{eqnarray}
%&\widehat{f(r)}_{-N_\theta/2} &= \widehat{f(r)}_{N_{\theta/2}} / 2 \\
%&\widehat{f(r)}_{N_\theta/2}  &= \widehat{f(r)}_{N_{\theta/2}} / 2
%\end{eqnarray}

%We have
%\begin{equation}
%f(r, \theta_\tau) = \frac{1}{2\pi} \sum_{k=-N_\theta/2}^{N_\theta/2} \widehat{f(r)}_k e^{ik\theta_\tau}.
%\end{equation}

%For more derivation, we use the following interpolant of $f(r, \theta)$:
%\begin{equation}
%f(r, \theta) = \frac{1}{2\pi} \sum_{k=-N_\theta/2}^{N_\theta/2} \widehat{f(r)}_k e^{ik\theta}.
%\end{equation}

%Then
%\begin{center}
%\begin{eqnarray}
% T_4&=& \int_a^b r^2\phi_p(r) \int_0^{2\pi} \frac{1}{2\pi} \sum_{k=-N_\theta/2}^{N_\theta/2} \widehat{f(r)}_k e^{ik\theta} e^{iq\theta} d\theta dr \\
%    &=& \int_a^b r^2\phi_p(r) \sum_{k=-N_\theta/2}^{N_\theta/2} \widehat{f(r)}_k \frac{1}{2\pi} \int_0^{2\pi}  e^{ik\theta} e^{iq\theta} d\theta dr \\
%    &=& \int_a^b r^2\phi_p(r) \sum_{k=-N_\theta/2}^{N_\theta/2} \widehat{f(r)}_k \delta_{k,q} dr\\
%    &=& \int_a^b r^2\phi_p(r) \widehat{f(r)}_q dr\\
%    &=& \sum_\sigma w_\sigma r_\sigma^2\phi_p(r_\sigma) \widehat{f(r_\sigma)}_q.
%\end{eqnarray}
%\end{center}
Then
\begin{center}
\begin{eqnarray}
 T_4&=& \int_a^b r^2\phi_p(r) \int_0^{2\pi} \frac{1}{2\pi} \sum_{k=-N_\theta/2+1}^{N_\theta/2} e^{ik\theta} \widehat{f(r)}_k e^{iq\theta} d\theta dr \\
    &=& \int_a^b r^2\phi_p(r) \sum_{k=-N_\theta/2+1}^{N_\theta/2} \widehat{f(r)}_k \frac{1}{2\pi} \int_0^{2\pi}  e^{ik\theta} e^{iq\theta} d\theta dr \\
    &=& \int_a^b r^2\phi_p(r) \sum_{k=-N_\theta/2+1}^{N_\theta/2} \widehat{f(r)}_k \delta_{k,q} dr\\
    &=& \int_a^b r^2\phi_p(r) \widehat{f(r)}_q dr\\
    &=& \sum_\sigma w_\sigma r_\sigma^2\phi_p(r_\sigma) \widehat{f(r_\sigma)}_q.
\end{eqnarray}
\end{center}


Since we have this relation $- T_1 - T_2 - T_3 = T_4$,

\begin{eqnarray} \label{bdyt}
&&  \sum_j \sum_k \hat{u}_{jk} \delta_{qk} \left(2 ({\bf M}_2)_{pj} + ({\bf M}_3)_{pj} \right) \\
&-& \sum_j \sum_k \hat{u}_{jk} \delta_{qk} \left[ b^2\phi_p(b)\frac{d}{dr}\phi_j(b) - a^2\phi_p(a)\frac{d}{dr}\phi_j(a) \right] \\
&-& \sum_j \sum_k \hat{u}_{jk} \delta_{qk} ({\bf M}_2)_{pj}     \\
&+& \sum_j \sum_k \hat{u}_{jk} \delta_{qk} k^2 ({\bf M}_1)_{pj} \\
&=& \frac{T_4}{2\pi}
\end{eqnarray}
\begin{eqnarray} \label{bdyt}
&&  \sum_j \sum_k \hat{u}_{jk} \delta_{qk} \left(2 ({\bf M}_2)_{pj} + ({\bf M}_3)_{pj} \right) \\
&-& \sum_j \sum_k \hat{u}_{jk} \delta_{qk} ({\bf M}_2)_{pj}     \\
&+& \sum_j \sum_k \hat{u}_{jk} \delta_{qk} k^2 ({\bf M}_1)_{pj} \\
&=& \frac{T_4}{2\pi} \\
&+& \sum_j \sum_k \hat{u}_{jk} \delta_{qk} \left[
b^2\phi_p(b)\frac{d}{dr}\phi_j(b) -
a^2\phi_p(a)\frac{d}{dr}\phi_j(a) \right]
\end{eqnarray}
\begin{eqnarray} \label{bdyt}
&&  \sum_j \sum_k \hat{u}_{jk} \delta_{qk}  (({\bf M}_3)_{pj} + ({\bf M}_2)_{pj} + k^2 ({\bf M}_1)_{pj}) \\
&=& \frac{T_4}{2\pi} \\
&+& \sum_j \sum_k \hat{u}_{jk} \delta_{qk} \left[
b^2\phi_p(b)\frac{d}{dr}\phi_j(b) -
a^2\phi_p(a)\frac{d}{dr}\phi_j(a) \right]
\end{eqnarray}
Since further issues are closely linked with implementation, that
will be dealt with next example.
