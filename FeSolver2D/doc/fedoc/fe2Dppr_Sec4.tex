\section{Numerical Methodology}

We applied a finite element methods as an deterministic numerical
solver for given ECG forward modeling problem. In this section we
give a brief review about our approach and describe a problem
setup of which the polynomial chaos solver is making use in
uncertainty quantification. In the end, we how the error size
decreases by the refinement of the torso data set and simplified
square data for verification of accuracy of the solver.

\subsection{Finite Element Methods}

The data set is a 2-dimensional slice of torso volume data. It is
discrtised by triangular elements having straight edges. We define
nodal finite elements over them by assigning additional nodal
points and defining corresponding shape functions to obtain
linear, quadratic, and cubic finite element approximation of given
problem. Table\ref{torsospec} is showing the number of nodes and
elements for the 3 kinds of approximations.

\begin{table}
\centering \caption {\label{torsospec} The number of nodes and
elements for each approximation.}
\begin{tabular}{|c||c|c|c|c|} \hline
    &Interior Nodes & Exterior Boundary $\Omega{1}$ & Inner Boundaries $\Omega{0}$& Elements \\ \hline
    Linear Elements     &545    &   53 & 60 &1203 \\ \hline
    Quadratic Elements&2293 &   106 & 120 &1203 \\ \hline
    Cubic Elements&5244 &159 & 180 &1203    \\ \hline
\end{tabular}
\end{table}



The shape functions of certain order are obtained by taking
products of the 1-dimensional Lagrange polynomials of the order.
By definition in \cite{Hughes}, we used 3,6, and 10 nodal points
and shape functions for linear, quadratic, and cubic
approximations, respectively.

The deterministic part of our ECG modeling is represented by the
following Laplace equation with conductivity $\sigma(\vx)$ on the
torso $\Omega$. To approximate numerically, we formulate the
method of weighted residual by incorporating the standard
Bubnov-Galerkin method \cite{Karniadarkis} which uses test
functions to be same as trial function in weak formulation.

{\bf DEFINED ALREADY}

\begin{eqnarray}
\bigtriangledown \cdot {\sigma(\vx) \bigtriangledown}u(\vx) & = & f(\vx)\\
u(\Omega_{0}) & = & \alpha(\Omega_{0}) \\
\bigtriangledown u(\Omega_{1}) & = & 0
\end{eqnarray}

where $\Omega_{0}$ is the interior boundary, corresponding to the
surface of the heart, and $\Omega_{1}$ is the exterior boundary,
corresponding to the surface of the thorax.


\subsection{Numerical Accuracy}

\begin{figure}[h]
\begin{center}
  \epsfig{file = Figures/conv3_89_1.eps, width = 8cm}
  \epsfig{file = Figures/conv10_89_1.eps, width = 8cm}
  \caption{\label{torsoerr} Decrease of errors with respect to mesh element size and order of approximation: The norm used are $L^{2}$ norms. 3 lines are showing convergence in different approximation orders in trial(basis) functions: Linear (Blue line), Quadratic (Red tiny dots), and Cubic (Green dots)}
\end{center}
\end{figure}

\begin{center}
\begin{tabular}{|c|c|c|} \hline
    Linear& Quadratic & Cubic \\ \hline \hline
    3.503e-1 & 1.4e-3 & 2.44e-5\\ \hline
    8.82e-2     & 2.0e-4 & 1.51e-6\\ \hline
    3.93e-2  & 1.0e-4 &    3.0e-7\\ \hline \hline
    1.9904& 3.0287&    4.0082\\ \hline
\end{tabular}
\begin{tabular}{|c|} \hline
    Resolution\\ \hline \hline
    1\\ \hline
    2\\ \hline
    3\\ \hline \hline
    Error Decrease  \\ \hline
\end{tabular}
\begin{tabular}{|c|c|c|} \hline
    Linear& Quadratic & Cubic \\ \hline \hline
    e-1 & e-3 & e-5\\ \hline
    e-2 & e-4 & e-6\\ \hline
    e-2 & e-4 & e-7\\ \hline \hline
    & &   \\ \hline
\end{tabular}
\end{center}




\begin{thebibliography}{9}

\bibitem{Hughes}Hughes, Thomas. J. R. \/
        {\bf The Finite Element Method, Linear Static and Dynamic Finite Element
        Analysis}.
        Prentice Hall, 1987.

\bibitem{Johnson}Johnson, Christopher R. \/
        {\bf Lecture note of Advanced Methods in Scientific
        Computing}.
        School of Computing, University of Utah, 2002.

\bibitem{Karniadarkis}Karniadakis, George Em, Sherwin, Spencer J. \/
        {\bf Spectral/Hp Element Methods for Cfd}.
        Oxford Univ Press, 1999.

\bibitem{Burnett}Burnett, David S. \/
        {\bf Finite Element Analysis, From Concepts to
        Applications}.
        Addison-Wesley Publishing Company, 1987.

\bibitem{Zienkiewicz}Zienkiewicz, O. C., Taylor, R. L. \/
        {\bf The Finite Element Method; The Basis}.
        Butterworth Heinemann, 2000.

\end{thebibliography}
