\section{Numerical Methodology}

We applied a finite element methods as an deterministic numerical
solver for given ECG forward modeling problem. In this section we
give a brief review about our approach and describe a problem
setup of which the polynomial chaos solver is making use in
uncertainty quantification. In the end, we how the error size
decreases by the refinement of the torso data set and simplified
square data for verification of accuracy of the solver.

\subsection{Finite Element Methods}

The data set is a 2-dimensional slice of torso volume data. It is
discrtised by triangular elements having straight edges. We define
nodal finite elements over them by assigning additional nodal
points and defining corresponding shape functions to obtain
linear, quadratic, and cubic finite element approximation of given
problem. Table\ref{torsospec} is showing the number of nodes and
elements for the 3 kinds of approximations.


The shape functions of certain order are obtained by taking
products of the 1-dimensional Lagrange polynomials of the order.
By definition of nodal approximation shown in \cite{Burnett87},
\cite{Hughes87}, \cite{Johnson02}, and \cite{Zienkiewicz00}, we
used 3,6, and 10 nodal points and shape functions for linear,
quadratic, and cubic approximations, respectively.


The deterministic part of our ECG modeling is represented by the
following Laplace equation with conductivity $\sigma(\vx)$ on the
torso $\Omega$ as shown in equation (\ref{detelliptic}). To
approximate numerically, we formulate the method of weighted
residual by incorporating the standard Bubnov-Galerkin method
\cite{Karniadarkis99} which uses test functions to be same as
trial function in weak formulation.

\begin{eqnarray}
\label{detelliptic}
\bigtriangledown \cdot {\sigma(\vx) \bigtriangledown}u(\vx) & = & f(\vx)\\
u(\Omega_{0}) & = & \alpha(\Omega_{0}) \\
\bigtriangledown u(\Omega_{1}) & = & 0
\end{eqnarray}
where $\Omega_{0}$ is the interior boundary, corresponding to the
surface of the heart, and $\Omega_{1}$ is the exterior boundary,
corresponding to the surface of the thorax.


\begin{table}
\caption {\label{torsospec} The number of nodes and elements for
each approximation.}
\begin{tabular}{|c||c|c|c|c|} \hline
    &Interior Nodes & Exterior Boundary $\Omega{1}$ & Inner Boundaries $\Omega{0}$& Elements \\ \hline
    Linear Elements     &545    &   53 & 60 &1203 \\ \hline
    Quadratic Elements&2293 &   106 & 120 &1203 \\ \hline
    Cubic Elements&5244 &159 & 180 &1203    \\ \hline
\end{tabular}
\end{table}

\subsection{Numerical Accuracy}

To verify the accuracy of finite element solver, we tested two
experiments which took various conditions into account. The first
one is to see if the solver can process different scales on the
choice of different mesh data. We applied the solver to realistic
torso data whose scale vary on $(-200,200)\times(-200,200)$. The
we set Dirichlet boundary conditions on all exterior and interior
boundaries. The second experiment is to see if the solver is
accurately approximating a known analytic solution. This checked
the convergence of error to the known solution by using
progressive mesh refinements in 3 different order of
approximations. The figure(\ref{torsoerr}) is showing the change
of resultant errors in each experiment:
\begin{figure}[h]
\begin{center}
  \epsfig{file = Figures/conv3_89_1.eps, width = 8cm}
  \epsfig{file = Figures/conv10_89_1.eps, width = 8cm}
  \caption{\label{torsoerr} Decrease of errors with respect to mesh element size
  and order of approximation: The norm used are $L^{2}$ norms.
  3 lines are showing convergence in different approximation orders in trial(basis) functions:
  Linear (Blue line), Quadratic (Red tiny dots), and Cubic (Green dots). They are the result of
  solving on realistic torso data on 3 different mesh refinements(Left) and on Neumann and Dirichlet
  boundary conditions on progressive refinements of square data set(Right).}
\end{center}
\end{figure}

Table(\ref{table_err1}) is explicitly showing the error of the
finite element solver to known analytic solution. We choose
$u(x,y) = \cos(c_1 x)\cos(c_2 y)$ to be an analytic solution. We
noticed the errors decrease for increase of order of approximation
and mesh refinements. And it is showing the slope of error
decrease in both experiments. The values are close to $1+$order of
approximation as expected in theory.
\begin{table}
\begin{center}
\caption{\label{table_err1} (Left)The errors of approximation of
finite solver to the known analytic solution, (Right)The slopes
which show the speed of convergence of each different order of
approximations and mesh refinements}
\begin{tabular}{|c||c|c|c|} \hline
    Mesh Resolution&Linear& Quadratic & Cubic \\ \hline \hline
    1&3.503e-1 & 1.4e-3 & 2.44e-5\\ \hline
    2&8.82e-2  & 2.0e-4 & 1.51e-6\\ \hline
    3&3.93e-2  & 1.0e-4 & 3.0e-7\\ \hline
\end{tabular}
\begin{tabular}{|c||c|c|c|} \hline
    Experiments&Linear& Quadratic & Cubic \\ \hline \hline
    Ex. 1&1.9904& 3.0287&   4.0082\\ \hline
    Ex. 2&1.9630& 2.9928&   4.0186\\ \hline
\end{tabular}
\end{center}
\end{table}
