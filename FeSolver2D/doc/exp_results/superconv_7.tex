
\documentclass[11pt, notitlepage,  letterpaper]{article}


%% Horizontal Lengths - Max 6.5
\setlength{\footskip}{0.5in} \setlength{\hoffset}{0in}
\setlength{\oddsidemargin}{-.2in}
\setlength{\evensidemargin}{-.2in}
%\setlength{\textwidth}{6.9in}
\setlength{\textwidth}{6.6in}

%% Vertical Lengths - Max 9.0
\setlength{\headheight}{0in} \setlength{\topskip}{0in}
\setlength{\voffset}{0in} \setlength{\topmargin}{0.5in}
\setlength{\textheight}{8.4in}

%\usepackage{fancyheadings}

\usepackage{url}
\usepackage[]{epsfig,amsmath}[]


%\pagestyle{headings}
%\pagestyle{fancy}
\pagenumbering{arabic}


%%%%%%%%%%%%%%%%%%%%%%%%%%%%%%%%%%%%%%%%%%%%%%%%%%%%%%%%%%%%%%%%%%%%%%
% New definitions and commands
\newtheorem{define}{Definition}[section]
\newtheorem{theorem}[define]{Theorem}
\newtheorem{question}[define]{Question}
\newtheorem{problem}[define]{Problem}


%**************************************


\begin{document}
%\title{
%{CS6962: Coursework 2}\\{Partitioning a Problem in Different
%Ways}\\{\large Paper Submission} }
%
%\vfill
\author{Seungkeol\ Choe \\ skchoe@cs.utah.edu}

\renewcommand{\today}{Apr 30th, 2004}

%\maketitle

%\begin{abstract}
%\end{abstract}

%\noinden {\bf Keywords: Spectral Element Methods, Poisson Equation}

%\tableofcontents

%-------------------------------------------------------------------------------
%\clearpage
\section{Convergence of FE solver and slopes}

%\smallskip %\smallskipamount
\medskip %medskipamount
%\bigskip %bigskipamount
%\vfill
\renewcommand{\arraystretch}{1.5}
%\centering

We examined a superconvergence for finite element approximation using linear/quadratic/cubic basis functions and obtained the following results:

\begin{figure}[h]
    \begin{center}
    \epsfig{file = superconv70.eps, width = 16cm}
    \caption{\label{scrvsol1}Convergence Test for the Method where N = 4,8,12,16,20, and 24 to the exact solution $u(x,y) = e^{- x^2 - y^2}$}
    \end{center}
\end{figure}

The slopes for convergence in terms of 2 kinds of norm are as follows:

\begin{center}
\begin{tabular}{|c||r|r|r|}  \hline
Norm  & Linear & Quadratic & Cubic\\ \hline \hline
$L^{\infty}$&1.7717&3.6939&3.9453 \\\hline
$L^2$&2.0243&3.0282&4.0368\\\hline
\end{tabular}
\end{center}

The example solution was
\begin{equation}
u(x,y) = e^{- x^2 - y^2}.
\end{equation}

Conductivity Matrix A was defined as:
\begin{eqnarray}
    A(1,1) &=& \cos^2(\pi r^2) + \sin^2(\pi r^2)\\
    A(1,2) &=& A(2,1) = \cos(\pi r^2) +  \sin(\pi r^2) + 2\\
    A(2,2) &=& (\sin(\pi r^2)+2)^2
\end{eqnarray}

We subdivide the domain more finely to obtained small size of errors:

\begin{figure}[h]
    \begin{center}
    \epsfig{file = superconv71.eps, width = 16cm}
    \caption{\label{scrvsol2}Continued Convergence Test N = 24,28,32,36, and 40 $u(x,y) = e^{- x^2 - y^2}$}
    \end{center}
\end{figure}

The following table shows slopes for convergence in terms of 2 kinds of norm:
\begin{center}
\begin{tabular}{|c||r|r|r|}  \hline
Norm  & Linear & Quadratic & Cubic\\ \hline \hline
$L^{\infty}$&1.8389&3.9372&3.9843 \\\hline
$L^2$&1.9992&3.0074&4.0099\\\hline
\end{tabular}
\end{center}

\newpage
This graph shows the result again which combines previous results.
\begin{figure}[h]
    \begin{center}
    \epsfig{file = superconv72.eps, width = 16cm}
    \caption{\label{scrvsol2} N = 4, 8, 12, 16, 20, 24,28,32 $u(x,y) = e^{- x^2 - y^2}$}
    \end{center}
\end{figure}

The following table shows slopes for convergence in terms of 2 kinds of norm:
\begin{center}
\begin{tabular}{|c||r|r|r|}  \hline
Norm  & Linear & Quadratic & Cubic\\ \hline \hline
$L^{\infty}$&1.4306&2.7583&4.0664 \\\hline
$L^2$&1.8903&3.0051&4.1248\\\hline
\end{tabular}
\end{center}


\newpage
All together : total 9 experiments.
\begin{figure}[h]
    \begin{center}
    \epsfig{file = superconv74.eps, width = 16cm}
    \caption{\label{scrvsol2} N = 4, 8, 12, 16, 20, 24,28,32, 36 $u(x,y) = e^{- x^2 - y^2}$}
    \end{center}
\end{figure}

The following table shows slopes for convergence in terms of 2 kinds of norm:
\begin{center}
\begin{tabular}{|c||r|r|r|}  \hline
Norm  & Linear & Quadratic & Cubic\\ \hline \hline
$L^{\infty}$&1.4616&2.7922&4.0640 \\\hline
$L^2$&1.8950&3.0055&4.1194\\\hline
\end{tabular}
\end{center}

\end{document}
