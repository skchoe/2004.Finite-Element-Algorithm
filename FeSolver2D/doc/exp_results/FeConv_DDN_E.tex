
\documentclass[11pt, notitlepage,  letterpaper]{article}


%% Horizontal Lengths - Max 6.5
\setlength{\footskip}{0.5in} \setlength{\hoffset}{0in}
\setlength{\oddsidemargin}{-.2in}
\setlength{\evensidemargin}{-.2in}
%\setlength{\textwidth}{6.9in}
\setlength{\textwidth}{6.6in}

%% Vertical Lengths - Max 9.0
\setlength{\headheight}{0in} \setlength{\topskip}{0in}
\setlength{\voffset}{0in} \setlength{\topmargin}{0.5in}
\setlength{\textheight}{8.4in}

%\usepackage{fancyheadings}

\usepackage{url}
\usepackage[]{epsfig,amsmath}[]


%\pagestyle{headings}
%\pagestyle{fancy}
\pagenumbering{arabic}


%%%%%%%%%%%%%%%%%%%%%%%%%%%%%%%%%%%%%%%%%%%%%%%%%%%%%%%%%%%%%%%%%%%%%%
% New definitions and commands
\newtheorem{define}{Definition}[section]
\newtheorem{theorem}[define]{Theorem}
\newtheorem{question}[define]{Question}
\newtheorem{problem}[define]{Problem}


%**************************************


\begin{document}
%\title{
%{CS6962: Coursework 2}\\{Partitioning a Problem in Different
%Ways}\\{\large Paper Submission} }
%
%\vfill
\author{Seungkeol\ Choe \\ skchoe@cs.utah.edu}

\renewcommand{\today}{Apr 30th, 2004}

%\maketitle

%\begin{abstract}
%\end{abstract}

%\noinden {\bf Keywords: Spectral Element Methods, Poisson Equation}

%\tableofcontents

%-------------------------------------------------------------------------------
%\clearpage
\section{Convergence of FE solver and slopes: with Conductivity defined Elementwise}

%\smallskip %\smallskipamount
\medskip %medskipamount
%\bigskip %bigskipamount
%\vfill
\renewcommand{\arraystretch}{1.5}
%\centering

We examined a superconvergence for finite element approximation using linear/quadratic/cubic basis functions and obtained the following results:


Conductivity Matrix A is defined to be constant and diagonal over all domain:

We subdivide the domain more finely to obtained small size of errors:

\begin{figure}[h]
    \begin{center}
    \epsfig{file = EwiseErrDD.eps, width = 16cm}
    \caption{\label{scrvsol2}Continued Convergence Test N = 4, 8, 12, 16, 24,28,32,36, and 40 $u(x,y) = e^{- x^2 - y^2}$} with Dirichlet Boundary of Square.
    \end{center}
\end{figure}

The following table shows slopes for convergence in terms of 2 kinds of norm:
\begin{center}
\begin{tabular}{|c||r|r|r|}  \hline
Norm  & Linear & Quadratic & Cubic\\ \hline \hline
$L^{\infty}$&2.0233236&3.9334217&3.7874339 \\\hline
$L^2$&1.9673284&2.9872180&4.0287991\\\hline
\end{tabular}
\end{center}

\newpage

This shows the decrease of error for F.E. Solver of Elliptic problem defined on a square with Mixed(Dirichlet and Neumann, not Robin) boundary conditions. 

Conductivity Matrix A is defined to be constant and diagonal over each element of domain:

We subdivide the domain more finely to obtained small size of errors:

\begin{figure}[h]
    \begin{center}
    \epsfig{file = EwiseErrDN.eps, width = 16cm}
    \caption{\label{scrvsol2}Continued Convergence Test N = 4, 8, 12, 16, 24,28,32,36, and 40 $u(x,y) = e^{- x^2 - y^2}$} with Dirichlet and Neumann Boundary of Square.
    \end{center}
\end{figure}

The following table shows slopes for convergence in terms of 2 kinds of norm:
\begin{center}
\begin{tabular}{|c||r|r|r|}  \hline
Norm  & Linear & Quadratic & Cubic\\ \hline \hline
$L^{\infty}$&2.1286392&3.2724651&3.8174268 \\\hline
$L^2$&1.8352189&2.9784264&4.0018499\\\hline
\end{tabular}
\end{center}
\end{document}
