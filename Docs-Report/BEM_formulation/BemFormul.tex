\documentclass[11pt,letterpaper]{article}

% Vertical Lengths - Max 9.0
\setlength{\voffset}{0in} \setlength{\topmargin}{.0in}
\setlength{\headheight}{0in} \setlength{\topskip}{-1.2in}
\setlength{\textheight}{9.0in} \setlength{\footskip}{-.0in}

% Horizontal Lengths - Max 6.5
\setlength{\hoffset}{0in} \setlength{\oddsidemargin}{-.2in}
\setlength{\textwidth}{6.9in} \setlength{\evensidemargin}{-.2in}

%\usepackage{fancyheadings}

\usepackage{url}
\usepackage[]{epsfig,amsmath}[]


\pagestyle{headings}
%\pagestyle{fancy}
\pagenumbering{arabic}

%%%%%%%%%%%%%%%%%%%%%%%%%%%%%%%%%%%%%%%%%%%%%%%%%%%%%%%%%%%%%%%%%%%%%%
% New definitions and commands
\newtheorem{define}{Definition}[section]
\newtheorem{theorem}[define]{Theorem}
\newtheorem{question}[define]{Question}
\newtheorem{problem}[define]{Problem}

\begin{document}

\title{{\bfseries Boundary Element Method Formulation for Annulus Domain}
}
For angular direction we have mode $k = 0, \cdots, N-1$, azimuthal
direction we have mode $p = 0, \cdots, M$.

We have bases for each of direction
\begin{align}
\{\psi(r) \}_{r=0}^M,\\
\{e^{ik\theta}\}_{k=0}^{N-1}.
\end{align}

Then in 2D circular domain, the solution $u$ satisfies following
\begin{equation}
L(u)(x,y) \equiv \frac{d^2}{dx^2} u(x) + \frac{d^2}{dy^2} u(y)-
f(x,y) = 0,
\end{equation}
for all $(x,y)$ in a circular domain with a hole in the middle.


We change the coordinate to polar so that the variables are $r,
\theta$. We denote $u(x, y), f(x, y)$ in polar coordinate as
$\Tilde{u}(r, \theta), \Tilde{f}(r, \theta)$ such that

\begin{equation}
\frac{\partial^2}{\partial r^2} \Tilde{u} +
\frac{1}{r}\frac{\partial}{\partial r} \Tilde{u} +
\frac{1}{r^2}\frac{\partial^2}{\partial \theta^2} \Tilde{u} =
\Tilde{f}
\end{equation}

To erase the fraction term $\frac{1}{r^2}$, we use the fact the
domain has hole at the center, then $r \ne 0$. This enables to
multiply $r^2$ at both sides as follows

\begin{equation}
r^2 \frac{\partial^2}{\partial r^2} \Tilde{u} + r
\frac{\partial}{\partial r} \Tilde{u} + \frac{\partial^2}{\partial
\theta^2} \Tilde{u} = r^2 \Tilde{f}
\end{equation}

 By using the above bases functions, we represent $u$ as
\begin{align}
\Tilde{u}(r, \theta) =
\sum_{p=0,q=0}^{M,N-1}\Hat{u}_{p,q}\psi_p(r) e^{iq\theta} \\
\end{align}

We have the following linear combination:
\begin{equation}
\left(r^2 \nabla^2 \Tilde{u}, \Tilde{\nu} \right) =
\left(r^2 \Tilde{f}, \Tilde{\nu}\right),
\end{equation}
for each $\Tilde{\nu} = $ which has same representation as
$\Tilde{u}$.

For any two $(p,q)$, $(j,k)$ components in basis $\{\psi_p(r)
e^{iq\theta}\}_{p=0, q=0}^{M, N-1}$, we can define components of
inner product block matrix:


\begin{align*}
\left( r^2 \nabla^2 \left[ \psi_p(r) e^{iq\theta} \right],
\psi_j(r)e^{ik\theta} \right)
& =\int_{0}^{2\pi}\int_{a}^{b}
r^2\frac{\partial^2}{\partial r^2} \left[ \psi_p(r)
e^{iq\theta}\right]
\left[ \psi_j(r)e^{ik\theta} \right] dr d\theta &\cdots T_1\\
& + \int_{0}^{2\pi}\int_{a}^{b} r\frac{\partial}{\partial r}\left[
\psi_p(r) e^{iq\theta}\right]
\left[ \psi_j(r)e^{ik\theta}\right] dr d\theta  &\cdots T_2\\
& + \int_{0}^{2\pi}\int_{a}^{b} \frac{\partial^2}{\partial
\theta^2} \left[ \psi_p(r) e^{iq\theta}\right] \left[
\psi_j(r)e^{ik\theta}\right] dr d\theta         &\cdots T_3
\end{align*}
\newline

\begin{align*}
T_1\cdots & \int_{0}^{2\pi}\int_{a}^{b}
r^2\frac{\partial^2}{\partial r^2} \left[ \psi_p(r)
e^{iq\theta}\right]
\left[ \psi_j(r)e^{ik\theta} \right] dr d\theta \\
=& \left[ \int_{a}^{b}r^2\frac{\partial^2}{\partial r^2} \left[
\psi_p(r) \psi_j(r) \right] dr \right] \left[ \int_{0}^{2\pi}
e^{iq\theta}e^{ik\theta}d\theta \right] \\
=& \left[ r^2 \psi_j(r)\frac{\partial}{\partial r} \psi_p(r)
]\|_{a}^{b} - \int_{a}^{b}\frac{\partial}{\partial r}\left[ r^2
\psi_j(r)\right] \frac{\partial}{\partial r} \psi_p(r) dr \right]
\delta_{kq} \\
=& \left[ r^2 \psi_j(r)\frac{\partial}{\partial r} \psi_p(r)
]\|_{a}^{b} - \int_{a}^{b} ( 2r \psi_j(r) \frac{\partial}{\partial
r} \psi_p(r) + r^2\frac{\partial}{\partial r} \psi_j(r)
\frac{\partial}{\partial r} \psi_q(r)dr )\right] \delta_{kq}
\end{align*}

\begin{align*}
T_2\cdots & \int_{0}^{2\pi}\int_{a}^{b} r\frac{\partial}{\partial
r}\left[ \psi_p(r) e^{iq\theta}\right]
\left[ \psi_j(r)e^{ik\theta}\right] dr d\theta\\
=& \left[ \int_{a}^{b} r\frac{\partial}{\partial r}\psi_p(r)
\psi_j(r) dr \right] \left[ \int_{0}^{2\pi} e^{iq\theta}  e^{ik\theta}  d\theta \right] \\
=& \left[ \int_{a}^{b} r\frac{\partial}{\partial r}\psi_p(r)
\psi_j(r) dr \right] \delta_{qk}
\end{align*}

\begin{align*}
T_3\cdots & \int_{0}^{2\pi}\int_{a}^{b} \frac{\partial^2}{\partial
\theta^2} \left[ \psi_p(r) e^{iq\theta}\right] \left[
\psi_j(r)e^{ik\theta}\right] dr d\theta \\
=& \left[ \int_{a}^{b} \psi_p(r) \psi_j(r) dr \right] \left[
\int_{0}^{2\pi} \frac{\partial^2}{\partial \theta^2} e^{iq\theta}
e^{ik\theta} d\theta \right] \\
=& \left[ \int_{a}^{b} \psi_p(r) \psi_j(r) dr \right] \delta_{qk} \\
\end{align*}
\end{document}
