\documentclass[11pt,letterpaper]{article}

\setlength{\oddsidemargin}{-0.06in}
\setlength{\evensidemargin}{0.06in} \setlength{\topmargin}{0.4in}
\setlength{\textwidth}{6.50in} \setlength{\textheight}{8.93in}
\raggedbottom \hfuzz=3pt

\usepackage[]{epsfig,amsmath}[]
\pagestyle{headings}

\begin{document}

\title{{\bfseries Question on using fft, ifft in code} }
%\maketitle
The right hand side of weak form is  \begin{equation} G(p,q)
\equiv \int_a^b \int_0^{2\pi} f(r, \theta)\phi_p(r) e^{iq\theta}
d\theta dr. \end{equation}

Let $V$ be the discrete sampling of above function $f(r,
\theta)$,then $V$ is 2x2 array of components

\begin{equation}
v_{\sigma,\tau} = f(r_\sigma, \theta_\tau)
\end{equation}
where $\sigma = 0, \cdots, N_r-1$ and $\tau=0,
\cdots, N_\theta-1$ with $N_r$ is number of quadrature points for
all elements and $N_\theta$ is number of modes in $\theta$
direction element.

For numerical evaluation, we use Gauss-Lobatto Quadrature formula
for $r$ direction integral and Fourier transform for $\theta$
direction integral.

Note that there're weight factors $\{w_\sigma\}_{\sigma =0}^{N_r}$
which are used for weighted sum of integrand on each
$\{r_\sigma\}$.


For fixed $r_0$, the discrete Fourier transform for $f(r_0,
\theta)$is defined by

\begin{equation} f(r_0, \theta_\tau) =
\sum_{k=-N_\theta/2+1}^{N_\theta/2} \widehat{f(r_0)}_k
e^{ik\theta_\tau} \end{equation}

where $k$ $\in \{-\frac{N_\theta}{2}+1, \cdots,
\frac{N_\theta}{2}\}$.

The flow I understood in last meeting is as follows:

\begin{eqnarray}
G(p, q) &=& \int_a^b r^2\phi_p(r) \widehat{f(r)} dr \sum_{k=-N_\theta/2+1}^{N_\theta/2} \delta_{kq} \\
        &=& \int_a^b r^2\phi_p(r) \widehat{f(r)} dr \\
        &=& \sum_\sigma w_\sigma r_\sigma^2\phi_p(r_\sigma) \widehat{f(r_\sigma)}
\end{eqnarray}

The \textbf{first question} is how to deal with array form of
$\widehat{f(r_\sigma)}$. Since given $\sigma $,
$\widehat{f(r_\sigma)}$ would be defined from the vector
$[f(r_\sigma, \theta_\tau]_{\tau=0}^{N_\theta-1}$  $G(p, q) $ will
be vector form. This phenomena happens the same in using ifft() as
follows:

In our solution form,

\begin{equation} u(r,\theta) = \sum_{j=0}^{N_r}
\sum_{k=-N_\theta/2+1}^{N_\theta/2} \hat{u}_{jk} \phi_j(r)
e^{ik\theta} \end{equation}, we can apply ifft to compute terms
having $e^{ik\theta}$ like this:

\begin{eqnarray}
u(r,\theta) &=& \sum_{j=0}^{N_r} \phi_j(r) \sum_{k=-N_\theta/2+1}^{N_\theta/2} \hat{u}_{jk} e^{ik\theta} \\
        &=& \sum_{j=0}^{N_r} \phi_j(r) N_\theta ifft(\hat{u}_{jk})\\
\end{eqnarray} here, since we can compute $ifft(\hat{u}_{jk})$
from the vector form $[\hat{u}_{jk}]_{k=-N_\theta/2 +
1}^{N_\theta/2}$ But this also produces vector form.

The \textbf{second question} is to verify Trefthan's definition of
discrete Fourier transform (dft/inverse dft pair). In my code that
implements dft/inverse dft, when I change scale term in front of
summation differently from his book, I could obtain similar result
to the result of matlab fft(), ifft(). I attach the code here.

In his book I omit $h$ in (3.2) and change $\frac{1}{2\pi}$ to
$\frac{1}{N_\theta}$ in (3.3). The result of this change is
different from that of matlab fft(), ifft() in that

\begin{itemize} \item The counting order \item sign of $Re(\hat
u)$ when $\theta \ne 0$.\end{itemize} which were talked in last
meeting.

I think in last meeting I couldn't get the point how to implement
what I learned. For example, the how to connect $k$ in

\begin{equation} M_3 + M_2 -k^2 M_1 \end{equation} with fft() in
RHS.

For this I decide to finish the write-up and verify all things.
This would be good to prevent unexpected things from coming out
during programming.


 \end{document}
