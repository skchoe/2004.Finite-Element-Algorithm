\documentclass[11pt,letterpaper]{article}

% Vertical Lengths - Max 9.0
\setlength{\voffset}{0in} \setlength{\topmargin}{.0in}
\setlength{\headheight}{0in} \setlength{\topskip}{-1.2in}
\setlength{\textheight}{9.0in} \setlength{\footskip}{-.0in}

% Horizontal Lengths - Max 6.5
\setlength{\hoffset}{0in} \setlength{\oddsidemargin}{-.2in}
\setlength{\textwidth}{6.9in} \setlength{\evensidemargin}{-.2in}

%\usepackage{fancyheadings}

\usepackage{url}
\usepackage[]{epsfig,amsmath}[]


\pagestyle{headings}
%\pagestyle{fancy}
\pagenumbering{arabic}

%%%%%%%%%%%%%%%%%%%%%%%%%%%%%%%%%%%%%%%%%%%%%%%%%%%%%%%%%%%%%%%%%%%%%%
% New definitions and commands
\newtheorem{define}{Definition}[section]
\newtheorem{theorem}[define]{Theorem}
\newtheorem{question}[define]{Question}
\newtheorem{problem}[define]{Problem}

\begin{document}

\title{{\bfseries Matlab implementation of double integral in Fourier method}
}

%\vfill
%\author{Seung-Keol Choe \thanks{Computational Engineering and Science program, University of Utah}\\
%        Mike Kirby \thanks{Assistant Professor, Scientific Computing and Imaging Institute, University of Utah}
%       }

%\renewcommand{\today}{Aug 28th, 2003}

%\maketitle

%\begin{abstract}
%This document is for specifying details of Spectral methods and
%Fourier method.
%\end{abstract}

%\tableofcontents
%\listoffigures
%\listoftables

%\clearpage
%--------------------------------------------------------------------------------------
%  SEC.1 A Steady-State Diffusion Problem
%\section {A Steady-State Diffusion Problem}
According to a report\cite{Karniadarkis}
\subsection {Diffusion Problem}
\subsection {Galerkin Method with Weak Solution}
\subsection {Boundary Conditions}


Double integral in Fourier method is defined as follows
\begin{equation}
G(p,q) \equiv \int_a^b \int_0^{2\pi} f(r, \theta) \psi_p(r)
e^{iq\theta} d\theta dr.
\\
\end{equation}

Let $V$ be the discrete sampling of above function $f(r, \theta)$,
then $V$ is 2x2 array of components
\begin{equation}
v_{s,t} = f(r_s, \theta_t)
\end{equation}
where $s = 0, \cdots, M-1$ and $t=0, \cdots, N-1$ with $M$ is
number of quadrature points for current element and $N$ is number
of elements in the circle.

For each $s = 0, \cdots, M-1$, the matlab fft($v_s^T$) $\equiv
\{v_{s,k}\}_{k=0}^{N-1}$ for vector $v_s^T$ is defined as
\begin{equation}
\mbox{fft}(v_s^T)_k \equiv v_{s,k} = \sum_{t=0}^{N-1} v_{s,t} e^{ik\theta_t} \\
\end{equation}
where $\{\theta_t\}_{t = 0}^{N-1} = \{0, \frac{2\pi}{N}, \cdots,
\frac{2\pi}{N}(N-1)\}$, and $k = 0, \cdots, N-1$.

Then we have the approximation of $G(p,q)$ based on Gauss-Lobatto
quadrature formula and approximation of integral by Trepezoidal
rule.
\begin{equation}
G(p, q) \approx \sum_{s=0}^{M-1} \omega_s \psi_p(r_s)
\sum_{t=0}^{N-1} v_{s,t} e^{iq\theta_t}.
\end{equation}
We obtain this by using fft() in matlab:
\begin{equation}
\sum_{s=0}^{M-1} \omega_s \psi_p(r_s) \mbox{fft}(v^T_{s})_q.
\end{equation}

Here's some questions to check this idea is logically correct.
\begin{itemize} \item Is integration just approximation
or exact evaluation? \item Is Trapezoidal rule right to be applied
to change from integral to summation? \item The integration
approximation needs usually dividing the summation by N. I think I
have to use $\frac{1}{N}$ in front of summation representation of
integral. \item In Fourier basis, we can substitute the exponent
from (-) to (+). Why are still use from 0 to N?
\end{itemize}


\end{document}
