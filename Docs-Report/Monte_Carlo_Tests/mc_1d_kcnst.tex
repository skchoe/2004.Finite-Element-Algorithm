\documentclass[11pt,letterpaper]{article}

% Vertical Lengths - Max 9.0
\setlength{\voffset}{0in} \setlength{\topmargin}{.0in}
\setlength{\headheight}{0in} \setlength{\topskip}{-2.2in}
\setlength{\textheight}{9.0in} \setlength{\footskip}{-.0in}

% Horizontal Lengths - Max 6.5
\setlength{\hoffset}{0in}
\setlength{\oddsidemargin}{-.2in}
\setlength{\textwidth}{6.9in}
\setlength{\evensidemargin}{-.2in}

%\usepackage{fancyheadings}

%\usepackage{url}
\usepackage[]{epsfig,amsmath}[]

%\pagestyle{headings}
\pagestyle{plain}
\pagenumbering{arabic}

%%%%%%%%%%%%%%%%%%%%%%%%%%%%%%%%%%%%%%%%%%%%%%%%%%%%%%%%%%%%%%%%%%%%%%
% New definitions and commands
\newtheorem{define}{Definition}[section]
\newtheorem{theorem}[define]{Theorem}
\newtheorem{question}[define]{Question}
\newtheorem{problem}[define]{Problem}


%%%%%%%%%%%%%%%%%%%%%%%%%%%%%%%%%%%%%%%%%%%%%%%%%%%%%%%%%%%%%%%%%%%%%%
% Start of document body
\begin{document}

\title{{\bfseries Results of  Monte Carlo Test for 1-dimensional Spectral Polynomial Methods}}

%\vfill
\author{Seungkeol Choe}
\renewcommand{\today}{Feb 12th, 2004}

\maketitle

\begin{abstract}
In this document we write the result history for the Monte-Carlo experiment on 
1-dimensional Spectral Polynomial Element methods. Throughout the updates, we try the simulation in various settings around the given Elliptic equation. 
\end{abstract}

%\tableofcontents
%\listoffigures
%\listoftables


%\clearpage
%----------------------------------------------------------------------

The overall problem in 1-dimensional partial differential equation is given by
\begin{problem}
\begin{equation*}
     \frac{d}{dx}\left[ \kappa(x)\frac{d}{dx}\right] u(x) = \sin(\pi x),
\end{equation*}
for all x in $[-1, 1]$.
The boundary conditions are as follows:
\begin{eqnarray}\label{bdycond}
    u(a) &=& {\mathcal G}_D \qquad \mbox{ : Dirichlet Condition}\\
    \frac{d}{dx}u(b) &=& 0\qquad \mbox{ : Neumann Condition}.
\end{eqnarray}
$\kappa(x)$ will be given in each experiment.
\end{problem}

\section {Result 1: Constant Case for $\kappa(x)$}

The problem setting for this experiment is followed by this table:
\begin{table}[h]
\begin{center}
\begin{tabular}{|l|l|} \hline
Offset of Dirichlet Condition& 100 \\ \hline
Number of iteration& 1000 \\ \hline
$\kappa(x) $& 5 \\ \hline
Sample Distribution& N(0,1) \\ \hline
Standard Deviation of Sample & 0.9526 \\ \hline
\end{tabular}
\end{center}
\end{table}

\clearpage

\begin{figure}[h]
\begin{center}
\epsfig{file = MeanVarCrv.eps, height = 6.5cm}
\epsfig{file = ErrToExactSol.eps, height = 6.5cm}
\caption{\label{MeanVarCrv} (Left)Distribution of results by Monte Carlo Simulation, (Right) The error graph: Difference between Mean of Monte-Carlo Solutions and Exact solution}
\end{center}
\end{figure}

By result, Figure \ref{MeanVarCrv} is showing the following information
\begin{itemize}
\item (Left Graph) 
	\begin{itemize}
	\item Mean of solutions(Blue line)
	\item Upper bdy of standard deviation from Mean(Red line)
	\item Lower bdy of standard deviation from Mean(Magenta line)
	\item 2 Differences of Red/Magenta lines to Blue line are approximately equal to the half of standard deviation of samples.
	\end{itemize}	

\item (Right Graph)
	\begin{itemize}
        \item A graph of curve of difference between Mean of solutions and Exact solution.
	\end{itemize}
\item Total Eerror: $L^{\infty}$
	\begin{itemize}
        \item 1.3157
	\end{itemize}
\end{itemize}

\section {Result 2: Heaviside Step Function Case for $\kappa(x)$}

\section {Result 3: High Curved Smooth Function Case for $\kappa(x)$}


\end{document}
