\subsection {Approximation of High order Polynomial solving
                                                1-D Poisson Equation}

In this section we construct a polynomial $P_n$ of order $n$
defined on $[0,1]$, which satisfies the following.
\begin{eqnarray*}
 P_n(0) = 0, &P_n(1) = 1 \\
 \frac{d^k}{dx^k}P_n(0) = 0, &\frac{d^k}{dx^k}P_n(1) = 0
\end{eqnarray*}
for all $k = 1, \cdots, n-2$. \\
Then for each $n$, we obtain a polynomial $P_n$ by solving a
system of linear equations having unique solution which determines
the set of coefficients of $P_n$. We will apply the spectral
polynomial solver to approximate the second derivative $Q_{n-2}$
of $P_n$.

Figure \ref{sol1} is showing some samples of solution of
order($n$) $21$.

\begin{problem}
\label{problem2}Consider the following differential equation for
$u(x)$ such that
\begin{equation*}
    \frac{d^2}{dx^2} u(x) = Q_{n-2},
\end{equation*}
for all $x$ in $[0, 1]$. Then the problem is to find approximation
$p(x)$ of $u(x)$ using spectral polynomial method.
\end{problem}

\noindent
%\begin{minipage}[b]{.46\linewidth}
\begin{figure}
  \centering%
  \epsfig{file = figs/0_sol21.eps, %
        height = 6cm}
  \caption{\label{sol1}Solution polynomial of order 21}
%\end{minipage}\hfill
\end{figure}


\subsubsection{Existence of Approximate Solution}

The Figure \ref{crvconvf1} and \ref{crvconvf2} are showing the
result of spectral polynomial method approximating the solution of
Problem \ref{problem2}. The maximum error value in Table
 \ref{crvconv1t} is showing the approximation is within numerically
exact solution tolerance.

\begin{figure}[h]
\begin{center}
\epsfig{file = figs/3_apcrv_app.eps, %
        height = 9cm}
\caption{\label{crvconvf1}Graph showing spectral approximation
satisfying Problem \ref{problem2}}
\end{center}
\end{figure}

\begin{figure}[h]
\begin{center}
\epsfig{file = figs/3_apcrv_err.eps, %
        height = 9cm}
\caption{\label{crvconvf2}Graph showing the error of approximation
in Figure \ref{crvconvf1}}
\end{center}
\end{figure}

\begin{table}[h]
\centering \caption{\label{crvconv1t} Specification of
                              Figure \ref{crvconvf1} and its error}
\begin{tabular}{|c|c|c|c|} \hline
Element Size &Num. of Element &Orders    &Err   \\ \hline \hline
$0.2$        &$5$             &$5, 5, 5, 5, 5$ &$5.7732e-015$ \\
\hline
\end{tabular}
\end{table}


\clearpage
%----------------------------------------------------------------------

\subsubsection{Convergence of Solution in Equidistance and
Uniformly Ordered Elements}

In this section we show the convergence of solutions obtained by
handling orders of basis on each element. We fix the element to be
same size(length) and divide the domain $[0, 1]$ by 5 elements.

Figure \ref{crvconvf3} is the result of convergence to
approximating to a solution of order $5$ in Problem
\ref{problem2}. It shows monotonic decreasing with same similar
slope until the order reaches from $1$ to $4$, and the slope get
stiff between order $4$ and $5$.

Figure \ref{crvconvf4} is that of solution of order $7$. In this
case, the error stops to decrease after the order is larger than
$7$. This part should be considered carefully and need to be made
sure the applicable range of the numerical method.

\begin{figure}[h]
\begin{center}
\epsfig{file = figs/31_equiconv_ord7.eps, %
        height = 9cm}
\caption{\label{crvconvf4}Graph showing convergence of order 7
problem}
\end{center}
\end{figure}

\begin{figure}[h]
\begin{center}
\epsfig{file = figs/explicitdd_od10_int5_10.eps, %
        height = 9cm}
\caption{\label{crvconvf4}Graph showing convergence of order
10(explicit) problem with 5/10 elements}
\end{center}
\end{figure}

%\clearpage
%----------------------------------------------------------------------

\subsubsection{Test of Convergence of Solution in Variable Ordered
Elements}

According to the idea that the solution need to be carefully
approximated specially in the center of the curve, we assign
different orders by the position of elements. I tested 2 cases.
The one is to variate 3 elements in center among 5 elements. The
other is to variate 1 element in the exact center of all elements.

Figure \ref{crvconvf5} is the first case with 3 different orders
at the $2$ ends of elements. Since it is based on the solution of
order $5$, the orders in 3 center elements moves from $1$ to $5$.

Figure \ref{crvconvf6} is the same as Figure \ref{crvconvf5}
except that it is based on solution of order 7 problem and the
orders at the center elements varies from 1 to 7.


\begin{figure}[h]
\begin{center}
\epsfig{file = figs/32_variconv_ordppp5.eps, %
        height = 9cm}
\caption{\label{crvconvf5}Graph showing convergence of order 5
problem}
\end{center}
\end{figure}

\begin{figure}[h]
\begin{center}
\epsfig{file = figs/32_variconv_ordppp7.eps, %
        height = 9cm}
\caption{\label{crvconvf6}Graph showing convergence of order 7
problem}
\end{center}
\end{figure}

Figure \ref{crvconvf7} and Figure \ref{crvconvf8} are the same as
\ref{crvconvf5} and \ref{crvconvf6} except that these have the
element that varies only a center element. The 2 different control
of orders on each element doesn't give out much different error
movement. This means choosing wise orders in each element can save
time of computing since the lower is the order, the faster does
the system solve.

\begin{figure}[h]
\begin{center}
\epsfig{file = figs/32_variconv_ordp5.eps, %
        height = 9cm}
\caption{\label{crvconvf7}Graph showing convergence of order 5
problem}
\end{center}
\end{figure}

\begin{figure}[h]
\begin{center}
\epsfig{file = figs/32_variconv_ordp7.eps, %
        height = 9cm}
\caption{\label{crvconvf8}Graph showing convergence of order 7
problem}
\end{center}
\end{figure}
