\documentclass[11pt, notitlepage,  letterpaper]{article}


%% Horizontal Lengths - Max 6.5
\setlength{\footskip}{0.5in}
\setlength{\hoffset}{0in}
\setlength{\oddsidemargin}{-.2in}
\setlength{\evensidemargin}{-.2in}
%\setlength{\textwidth}{6.9in}
\setlength{\textwidth}{6.6in}

%% Vertical Lengths - Max 9.0
\setlength{\headheight}{0in} \setlength{\topskip}{0in}
\setlength{\voffset}{0in} \setlength{\topmargin}{0.5in}
\setlength{\textheight}{8.4in}

%\usepackage{fancyheadings}

\usepackage{url}
\usepackage[]{epsfig,amsmath}[]


%\pagestyle{headings}
%\pagestyle{fancy}
\pagenumbering{arabic}


%%%%%%%%%%%%%%%%%%%%%%%%%%%%%%%%%%%%%%%%%%%%%%%%%%%%%%%%%%%%%%%%%%%%%%
% New definitions and commands
\newtheorem{define}{Definition}[section]
\newtheorem{theorem}[define]{Theorem}
\newtheorem{question}[define]{Question}
\newtheorem{problem}[define]{Problem}


%**************************************

\begin{document}
%\title{{\normalsize Fall 2003, Semester research;}\\
\title{
{Solving One-dimensional Forward Problems }\\{Using Spectral Element Methods}}

\vfill
\author{Seungkeol\ Choe \thanks{Computational Engineering and Science Program, University of Utah}}

\renewcommand{\today}{Jan 7th, 2004}

\maketitle

\begin{abstract}
In this report, we present a spectral polynomial method for solving a Poisson equation with Dirichlet and Neumann boundary conditions respectively on a one dimensional compact interval.
We can control the number, size of elements, and order of approximating polynomials to obtain accurate solution with faster convergence than its special case, the classical finite element method.
\end{abstract}

%\noinden {\bf Keywords: Spectral Element Methods, Poisson Equation}

\tableofcontents

%-------------------------------------------------------------------------------
\clearpage

\section{Introduction}
\section {A Steady-State Diffusion Problem}
According to a report\cite{Karniadarkis}
\subsection {Diffusion Problem}
\subsection {Galerkin Method with Weak Solution}
\subsection {Boundary Conditions}



\section{Spectral Polynomial Elements Method on an Interval}
\section {Spectral Polynomial Methods in 1 Dimensional Space}

Our problem is the Poisson equation
\begin{equation}
\label{poisson1} L(u) \equiv  \nabla^2 u - f = 0,
\end{equation}
for all $x \in \Omega$. In one dimensional case, \ref{poisson1} is
written as
\begin{equation}
\label{poisson2} L(u)(x) \equiv \frac{d^2}{dx^2} u(x) - f(x) = 0,
\end{equation}
for all $x$ in $[a, b]$.

We solve this equation in weak sense. That is to say, we define a
functional $\left( \cdot \right) : C^0 \rightarrow \Re$ such that
\begin{equation}
\left(\nabla^2 u, \nu \right) = \int_{\Omega} \frac{d^2}{dx^2}
u(x) \cdot \nu(x) d\mu(x),
\end{equation}
for each $\nu$ in $C^0$, where $C^0$ is a set of continuous
functions. Then we find the solution u by solving the equation as
follows:

\begin{equation} \label{weakabs}
\left(\nabla^2 u, \nu \right) = \left(f, \nu\right),
\end{equation}
for each $\nu$ in $C^0$.

%--------------------------------------------------------------------------
\subsection {Basis Functions}

The spectral approximation of solution $u$ is generally
represented as
\begin{equation}\label{genrep}
u(x) = \sum_{i=0}^{N_{dof-1}} \hat u_i(x)\Psi_i(x)
\end{equation}
on $[a, b]$. To construct this the global basis functions
$\{\Psi_i(x)\}_{i=0}^{N_{dof-1}}$, each $\Psi$ is represented by
the linear combination of local basis functions $\psi_i$ on each
element in $[a, b]$, say $\Omega^e$.

We define a basis functions ${\psi_i}$ on $\Omega^{st}$ to be a
real valued function with the Legendre polynomial $\{P_i^{1,1}\}$
as follows:
\begin{equation}\label{locbasis}
  \psi_i(\xi) =\large \{
  \begin{array}{ll}
  \frac{1-\xi}{2}, & i=0 \\
  \frac{1+\xi}{2}, & i=1 \\
  \left(\frac{1-\xi}{2}\right)\left(\frac{1+\xi}{2}\right)P_{i-2}^{1,1}(\xi),
  &i\ge 2
  \end{array}
\end{equation}
for all $\xi$ in $[-1, 1]$.

Then on a single standard element $\Omega^{st}$, the approximation
$u(\xi)$ is represented as
\begin{equation}\label{locrep}
u(\xi) = \sum_{i=0}^{N^e} \hat u_{i}^{e}\psi_i(\xi),
\end{equation}
for $\xi$ in $\Omega^{st}$.

%--------------------------------------------------------------------------
\subsection {Spectral Polynomial Method in A Element}


We apply the basis representation \ref{locrep} to weak formulation
\ref{weakabs} with the same test function $\{\psi_q\}$, then we
obtain the following:
\begin{equation}\label{locmat}
 \sum_{p=0}^{N^e} \hat u_p^e \left(\nabla^2 \psi_p, \psi_q \right)
= \left(\nabla^2 \sum_{p=0}^{N^e} u_p^e \psi_p, \psi_q\right) =
\left( f, \psi_q \right)
\end{equation}
for $q = 0, \cdots, N^e$.

With this relation we can setup a system of linear equations for
the coefficient $\{\hat u_p^e\}_{p=0}^{N^e}$ with $N^e+1 \times
N^e+1$ matrix $\mathbf{L}_{N^e}$ defined as follows:
\begin{equation}\label{loceqn}
    \mathbf{L}_{N^e} \cdot \mathbf{\hat u} = \mathbf{f},
\end{equation}
where
\begin{eqnarray}\label{locdefs}
\mathbf{L}_{N^e}(p,q) &=
\int_{\Omega^e}\frac{d^2}{d\xi^2}\psi_p(\xi)\psi_q(\xi) d\xi, \\
\mathbf{\hat u} &= \left[\hat u_p\right]_{p=0}^{N^e}, \\
\mathbf{f} &= \left[ \int_{\Omega^e} f(\xi) \psi_q(\xi) d\xi
\right]_{q=0}^{N^e}.
\end{eqnarray}

\subsection {H/P Refinement using Global Assembly}


\section{Experiment Results}
In section 1, 2, we present the result of convergence in both h
refinement and p refinement with the following steady-state
Poisson differential equation:
\begin{equation*}
    \frac{d^2}{dx^2} u(x) = \sin(\pi x),
\end{equation*}
for all x in $[0, 1]$.


\subsection {H-Convergence of 1-D Spectral Method}

This test is to validate the relation between size of element and
the accuracy of approximation. We apply equidistance element and
investigate the movement of error scale. As shown in Figure
\ref{sinDDhconv} and \ref{sinDNhconv}, the smaller are the
elements, the more exact is the solution. Moreover by testing with
different order of basis, we also could see the fact that the
higher are orders, the faster do they converge.


\begin{itemize}

\item Dirichlet-Dirichlet Case
\begin{figure}[h]
\begin{center}
\epsfig{file = figs_dd/sinDDhconv.eps, %
        height = 9cm}
\caption{\label{sinDDhconv}Graph showing change of errors by the
increase of the number of elements: Dirichlet-Dirichlet}
\end{center}
\end{figure}

\begin{table}[h]
\centering \caption{\label{hconv1t} Specification of
                              Figure\ref{sinDDhconv} and their errors}
\begin{tabular}{|c|c|c|} \hline
Polynomial order&Error&Slope   \\ \hline \hline
    3&$1.1940e-012$ &$3.9946$ \\ \hline
    4&$2.2204e-015$ &$4.9875$ \\ \hline
    5&$2.4147e-015$ &$5.9839$ \\ \hline
\end{tabular}
\end{table}

\item Dirichlet-Neumann Case


\begin{figure}[h]
\begin{center}
\epsfig{file = figs_dn/sinDNhconv.eps, %
        height = 9cm}
\caption{\label{sinDNhconv}Graph showing change of errors by the
increase of the number of elements: Dirichlet-Dirichlet}
\end{center}
\end{figure}

\begin{table}[h]
\centering \caption{\label{hconv2t} Specification of
                              Figure\ref{sinDNhconv} and their errors}
\begin{tabular}{|c|c|c|} \hline
Polynomial order&Error&Slope   \\ \hline \hline
    3&$1.1620e-012$ &$4.0024$ \\ \hline
    4&$4.6629e-014$ &$4.9877$ \\ \hline
    5&$9.7367e-014$ &$5.9775$ \\ \hline
\end{tabular}
\end{table}


\end{itemize}



\subsection {P-Convergence of 1-D Spectral Method}



\begin{itemize}

\item Dirichlet-Dirichlet Case
\begin{figure}[h]
\begin{center}
\epsfig{file = figs_dd/sinDDpconv.eps, %
        height = 9cm}
\caption{\label{sinDDpconv}Graph showing change of errors by the
increase of the number of elements: Dirichlet-Dirichlet}
\end{center}
\end{figure}

\begin{table}[h]
\centering \caption{\label{pconv1t} Specification of
                              Figure\ref{sinDDpconv} and their errors}
\begin{tabular}{|c|c|} \hline
Element Size&Error   \\ \hline \hline
    0.2&$7.7716e-016$  \\ \hline
    0.1&$1.1796e-016$  \\ \hline
\end{tabular}
\end{table}

\item Dirichlet-Neumann Case


\begin{figure}[h]
\begin{center}
\epsfig{file = figs_dn/sinDNpconv.eps, %
        height = 9cm}
\caption{\label{sinDNpconv}Graph showing change of errors by the
increase of the number of elements: Dirichlet-Neumann}
\end{center}
\end{figure}

\begin{table}[h]
\centering \caption{\label{pconv2t} Specification of
                              Figure\ref{sinDNpconv} and their errors}
\begin{tabular}{|c|c|} \hline
Element Size&Error   \\ \hline \hline
    0.2&$8.3267e-016$  \\ \hline
    0.1&$6.6613e-016$  \\ \hline
\end{tabular}
\end{table}


\end{itemize}

%\{ {\it  sp1d\_ch3\_2.tex} \}

\begin{figure}[h]
    \begin{center}
    \epsfig{file = Doc-Report_Fwd1D/figs_dn/ScrvO_13.eps, width = 5cm}
    \caption{\label{scrvsol1}Example of a curve satisfing conditions (\ref{pois_scrv1}), with polynomial order $n=9$}.
    \end{center}
\end{figure}

\subsubsection {High order Polynomial Solution and Its Convergence}

In this section we construct a polynomial $P_n$ of order $n$ defined on $[0,1]$, which satisfies the following.
\begin{eqnarray}
\label{pois_scrv1}
    P_n(0) = 0, &P_n(1) = 1 \\
    \frac{d^k}{dx^k}P_n(0) = 0, &\frac{d^k}{dx^k}P_n(1) = 0
\end{eqnarray}
for all $k = 1, \cdots, n-2$. \\
For each $n$, we obtain a polynomial $P_n$ by solving a system of
linear equations that determines the coefficients of $P_n$. We
apply the spectral polynomial solver to approximate the second
derivative $Q_{n-2}$ of $P_n$. The numerical and exact solutions
by the solver we developed is shown in figure (\ref{scrvsol1}).



\begin{problem}
Consider the following differential equation for $u(x)$ such that
\begin{equation}
\label{poi_poly1}
    \frac{d^2}{dx^2} u(x) = Q_{n-2},
\end{equation}
for all $x$ in $[0, 1]$ with the boundary condition defined in
equation (\ref{pois_scrv1}). Approximate $u(x)$ using spectral
polynomial method.
\end{problem}

Note that the accuracy of the interpolation satisfying equations
(\ref{pois_scrv1}) is dependent on the stability of the matrix
defining the coefficients of interpolants. We used the Legendre
basis functions because they are known to be more stable than
monomials. Despite this choise, interpolation error is nearly
$e^{-13}$. This results in the same amount of convergence error in
p-type extension mode shown in right of Figure (\ref{ScrvconvDN1})
and Table (\ref{hconv2t1}).

\begin{enumerate}

\item {Convergence h-type extension for equation (\ref{poi_poly1})}
Examining the equation (\ref{pois_sin1}), in Figure
(\ref{ScrvconvDN1}), we observe that the error with respect to
$L^{\infty}$ of the discrete solution to the equation is
exponentially convergent with respect to the size of element. This
verifies the Log-Log scale of relation of theory
(\ref{hrelation}).
\item {Convergence p-type extension for equation (\ref{poi_poly1})}
This semi-Log scale plot also shows the exponential convergence of
p-type extension of trial functions. Note that we approximate the
finite order of the polynomials. So there exists the lowest order
$P_l$ that approximates with trial functions of order $P$ which $P
> P_l$ should shows the same convergence as the case using trial
functions of order $P_l$. In right of Figure (\ref{ScrvconvDN1}),
we observe that the convergence is staying on approximation error
which theoretically should be machine precision.
\end{enumerate}

\begin{figure}[h]
\begin{center}
\epsfig{file = Doc-Report_Fwd1D/figs_dn/ScrvHconv.eps, width = 8.3cm}
\epsfig{file = Doc-Report_Fwd1D/figs_dn/ScrvPconv.eps, width = 8.3cm}
\caption{\label{ScrvconvDN1}
(Left) Convergence with respect to discrete $L^{\infty}$ norm as a function of  element
size. This test is performed using the h-type extension with fixed
polynomial order 3, 4, and 5 respectively. Error on the Log-Log
axis demonstrates the algebraic convergence of the h-type
extension.
(Right) Convergence w.r.t. $L^{\infty}$ norm as a function of size
of polynomial order in semi-Log plot. It shows the exponential
convergence of p-type extension for smooth solution. The two tests
are performed for p-type extension with element lengths of $0.2$
and $0.1$. }
\end{center}
\end{figure}

\begin{table}[h]
\centering \caption{\label{hconv2t1} This table shows the
convergence of h-type resolution control done above Figure
(\ref{ScrvconvDN1}). Note that the slopes of each order $P$ is
$P+1$ }
\begin{tabular}{|c|c|c|} \hline
    Polynomial order&Error($L^{\infty}$)&Slope   \\ \hline \hline
    3&$7.5530e-011$ &$3.9908$ \\ \hline
    4&$4.5619e-013$ &$4.6486$ \\ \hline
    5&$4.1855e-013$ &$5.7218$ \\ \hline
\end{tabular}
\hspace{.5in}
\begin{tabular}{|c|c|} \hline
    &\multicolumn{1}{|c|}{Error}\\
    \raisebox{0.5\baselineskip}%
    {Element Size}&($L^{\infty}$) \\ \hline \hline
    0.2&$3.0431e-013$  \\ \hline
    0.1&$3.1186e-013$ \\ \hline
\end{tabular}
\end{table}


\section{Conclusion}
%\clearpage
%\{ {\it  sp1d\_ch4\_1.tex} \}

Throughout this project, we looked into the theory of Spectral Polynomial Element Method and its solvability. We also investigated the feature of convergence in the viewpoint of h/p convergence property in comparison to classical finite element method. By using Galerkin method, we could incorporate the weak solution and get the problem to be changed to system of linear equations which we can solve it by computer.

By implementing the Spectral element solver for one dimensional
Poisson equation having Dirichlet and Neumann boundary conditions,
we could experiment all the theory with various specific cases of
high-order solutions which were hard to get acceptable convergence
in a given time and resolution of domain.

For the future study, we would like deal with problems regarding:

\begin{itemize}
\item
developing the solver to be applicable to problems defined on special domain by expanding the solver to multi-dimensional one.
\item
based on the theory of Spectral Element Method, doing research about various natural phenomena which we can understand it by a specific mathematical modelling.
\item
utilizing the method in the problem which is ill-posed. By using certain technique that we can approximate the solution, we can also apply this method to the problem and compare with other method in that situation.
\end{itemize}


%------------------------------------------------------------------------------
%   BIBLIOGRAPHY
%\clearpage

%\{ {\it  sp1d\_chbib.tex} \}

\begin{thebibliography}{9}

\bibitem{Karniadarkis}{\bf Spectral/Hp Element Methods for Cfd}
        George Em Karniadakis, Spencer J. Sherwin, \/
        Oxford Univ Press, 1999.

\bibitem{Trefethen}{\bf Spectral Methods in MATLAB}
        Lloyd N. Trefethen, \/
        Society for Industrial and Applied Mathematics, 2001.

\bibitem{Johnson}{\bf Lecture note of Advanced Methods in Scientific Computing}
        Christopher R. Johnson, \/
        School of Computing, University of Utah, 2002.


\end{thebibliography}



%-------------------------------------------------------------------------------
%   APPENDIX
%S
%%\clearpage
\appendix
\section {Gaussian Quadrature Formula}



\end{document}
