\documentclass[11pt, notitlepage,  letterpaper]{article}


%% Horizontal Lengths - Max 6.5
\setlength{\footskip}{0.5in}
\setlength{\hoffset}{0in}
\setlength{\oddsidemargin}{-.2in}
\setlength{\evensidemargin}{-.2in}
%\setlength{\textwidth}{6.9in}
\setlength{\textwidth}{6.6in}

%% Vertical Lengths - Max 9.0
\setlength{\headheight}{0in} \setlength{\topskip}{0in}
\setlength{\voffset}{0in} \setlength{\topmargin}{0.5in}
\setlength{\textheight}{8.4in}

%\usepackage{fancyheadings}

\usepackage{url}
\usepackage[]{epsfig,amsmath}[]


%\pagestyle{headings}
%\pagestyle{fancy}
\pagenumbering{arabic}


%%%%%%%%%%%%%%%%%%%%%%%%%%%%%%%%%%%%%%%%%%%%%%%%%%%%%%%%%%%%%%%%%%%%%%
% New definitions and commands
\newtheorem{define}{Definition}[section]
\newtheorem{theorem}[define]{Theorem}
\newtheorem{question}[define]{Question}
\newtheorem{problem}[define]{Problem}


%**************************************


\begin{document}
\title{
{Solving Two-dimensional Forward Problems on an Annulus}\\{Using
Spectral Fourier Methods}}

\vfill
\author{S.\ Choe \\ skchoe@cs.utah.edu \and R.\ M.\ Kirby \\ kirby@cs.utah.edu}

\renewcommand{\today}{Jan 30th, 2004}

\maketitle

\begin{abstract}
In this report, we present a spectral Fourier method for solving a
Poisson equation with Dirichlet and Neumann boundary conditions
respectively on an annulus domain. By using the tensor product
between spectral and Fourier bases, we could apply the high-order
method on the annulus domain. The fast Fourier transform allowed
the rapid convergence on the radius to be maintained to all
angular direction on the annulus.
\end{abstract}

%\noinden {\bf Keywords: Spectral Element Methods, Poisson Equation}

\tableofcontents

%-------------------------------------------------------------------------------
\clearpage

\section{Introduction}
\input{sp15d_ch1}

\section{The Numerical Solution of Poisson Equation on a 2-dimensional Annulus}
%\{ {\it  sp1d\_ch2.tex} \}

\subsubsection{Poisson Equation in Polar Coordinates and Basis Functions}

We formulate the Generalized Poisson problem on an annulus $[a, b]\times[0, 2\pi]$, $a > 0$ under the periodic solution $u$ as follows:
\begin{eqnarray}\label{genpois}
-\left[\frac{\partial}{\partial r} (\sigma(r,\theta) \frac{\partial}{\partial r}) + \frac{1}{r} (\sigma(r,\theta) \frac{\partial}{\partial r}) + \frac{1}{r^2}\frac{\partial}{\partial \theta} (\sigma(r,\theta)  \frac{\partial}{\partial \theta})\right] u(r, \theta) = f(r, \theta),\\
\mbox{with periodicity of }u, \;\;\; u(r,0) = u(r,2\pi),
\end{eqnarray}
where $r \in [a, b]$ and $\theta \in [0, 2 \pi]$.

The boundary conditions for this domain is given by:
\begin{equation}
u(a,\theta) = {\mathcal G}_D(\theta), \hspace{1in} \frac{\partial}{\partial r} u(b,\theta) = {\mathcal G}_N(\theta),
\end{equation}
where $\theta \in [0,2\pi]$.

\vspace{0.1in}
The representation of approximation of $u$ is guaranteed by Weierstrass theorem:
\begin{equation}\label{truncapp}
u(r,\theta) = \sum_{j=0}^{N_r} \sum_{k=-N_\theta/2+1}^{N_\theta/2} \hat{u}_{jk} \phi_j(r) e^{ik\theta},
\end{equation}
where $r \in [a, b]$ and $\theta \in [0, 2 \pi]$ for the global degree of freedom $N_r$ and $N_{\theta}$ on $\hat{u}_{jk}$'s.

\vspace{0.1in}

The basis functions $\{\phi_j\}_{j=0}^{N_r}$ shown at linear span
(\ref{truncapp}) are defined as modified Jacobi polynomials
defined in \cite{Karniadarkis}.

\vspace{0.1in}
As a review of discrete Fourier transform in $N$-point grid described in \cite{Trefethen}, the formula for the discrete Fourier transform for $\{v_j\}$ is
\begin{equation}
\hat{v}_k = h \sum_{j=1}^{N} e^{-ikx_j}v_j, \;\; k = -\frac{N}{2}+1, \ldots , \frac{N}{2},
\end{equation}
where $x_j = j\frac{2\pi}{N}$ and the inverse discrete Fourier transform for $\{\hat{v}_k\}$ is given by
\begin{equation}
v_j = \frac{1}{2\pi}\sum_{k = -N/2+1}^{Nr/2}e^{ikx_j}\hat{v}_k,\;\; j = 1, \ldots, N.
\end{equation}


\subsubsection{Formulation of Spectral Polynomial and Fourier Methods}

In this project, we assume the conductivity term $\sigma$ in
equation (\ref{genpois}) to be only dependent on variables showing
the radius domain as we multiply $r^2$ in each side of equation
(\ref{genpois}). Then the Poisson equation in polar coordinate is
as follows:
\begin{eqnarray}
-\left[r^2 \frac{\partial}{\partial r} (\sigma(r) \frac{\partial}{\partial r}) + r \sigma(r) \frac{\partial}{\partial r} + \sigma(r) \frac{\partial^2}{\partial \theta^2}\right] u(r, \theta) = r^2 f(r, \theta).
\end{eqnarray}

\vspace{0.1in}

We utilize the Galerkin method, with test functions of the form:
\begin{equation}
\phi_p(r) e^{iq\theta}, \hspace{.5in} p=0,\ldots,N_r, \;\; q = -\frac{N_\theta}{2}+1,\ldots,\frac{N_\theta}{2}.
\end{equation}

\vspace{0.1in}

The weak form of the equation is as follows:
\begin{equation}\label{galeqn}
-\langle  r^2 \frac{\partial}{\partial r} (\sigma \frac{\partial}{\partial r}u) + r \sigma \frac{\partial}{\partial r}u + \sigma \frac{\partial^2}{\partial \theta^2}u, \phi_p e^{iq\theta} \rangle = \langle r^2 f, \phi_p e^{iq\theta} \rangle.
\end{equation}

\vspace{0.1in} Define $T_i, i = 1,\ldots,4$ as follows
\begin{eqnarray}
T_1 &=& \int_{0}^{2\pi} \int_{a}^{b} \phi_p e^{iq\theta} r^2 \frac{\partial}{\partial r}\left[\sigma(r) \frac{\partial}{\partial r}u(r, \theta)\right] dr d\theta,\\
T_2 &=& \int_{0}^{2\pi} \int_{a}^{b} \phi_p e^{iq\theta} r \sigma(r) \frac{\partial}{\partial r}u(r, \theta) dr d\theta,\\
T_3 &=& \int_{0}^{2\pi} \int_{a}^{b} \phi_p e^{iq\theta} \sigma(r) \frac{\partial^2}{\partial \theta^2}u(r, \theta) dr d\theta,\\
\mbox{and} \hspace{.5in}T_4 &=& \int_{0}^{2\pi} \int_{a}^{b} \phi_p e^{iq\theta} r^2 f(r, \theta) dr d\theta
\end{eqnarray}

Then equation (\ref{galeqn}) becomes:
\begin{equation}\label{smpeqn}
- T_1 - T_2 - T_3 = T_4.
\end{equation}

We obtain boundary terms by integrating by parts on $T_1$ as
follows:
\begin{eqnarray}
T_1 &=& \int_0^{2\pi}e^{iq\theta} \left[ r^2\sigma(r)\frac{\partial}{\partial r}u(r,\theta)\phi_p(r)\right]_a^b d\theta \\
&-&2 \int_0^{2\pi}\int_a^b e^{iq\theta} r \sigma(r)\frac{\partial}{\partial r}u(r,\theta)\phi_p(r)drd\theta \\
&-& \int_0^{2\pi}\int_a^b e^{iq\theta} r^2\sigma(r)\frac{\partial}{\partial r}u(r,\theta) \frac{d}{dr}\phi_p(r)drd\theta.
\end{eqnarray}

Then the right hand side of (\ref{smpeqn}) becomes
\begin{eqnarray}\label{rhseqn1}
- T_1 - T_2 - T_3 &=& - \int_0^{2\pi}e^{iq\theta} \left[ r^2\sigma(r)\frac{\partial}{\partial r}u(r,\theta)\phi_p(r)\right]_a^b d\theta \\
&+& \int_0^{2\pi}\int_a^b e^{iq\theta} r \sigma(r) \frac{\partial}{\partial r}u(r,\theta)\phi_p(r)drd\theta \\
&+& \int_0^{2\pi}\int_a^b e^{iq\theta} r^2 \sigma(r) \frac{\partial}{\partial r}u(r,\theta) \frac{d}{dr}\phi_p(r)drd\theta \\
&-& \int_0^{2\pi}\int_a^b e^{iq\theta} \sigma(r) \frac{\partial^2}{\partial \theta^2}u(r, \theta) \phi_p(r) dr d\theta
\end{eqnarray}

Using (\ref{truncapp}) and the orthogonal properties of
$\{e^{ik\theta}\}$, $k =
-\frac{N_\theta}{2}+1,\ldots,\frac{N_\theta}{2}$, we can simplify
(\ref{rhseqn1}) to obtain:

\begin{eqnarray}\label{rhseqn2}
- T_1 - T_2 - T_3 &=& - \int_0^{2\pi}e^{iq\theta} \left[ r^2\sigma(r)\frac{\partial}{\partial r}u(r,\theta)\phi_p(r)\right]_a^b d\theta \\
&+& 2\pi \sum_{j=0}^{N_\theta} \hat{u}_{jq} \int_a^b r \sigma(r) \frac{d}{dr} \phi_j(r) \phi_p(r) dr \\
&+& 2\pi \sum_{j=0}^{N_\theta} \hat{u}_{jq} \int_a^b r^2 \sigma(r) \frac{d}{dr} \phi_j(r) \frac{d}{dr}\phi_p(r) dr \\
&+& 2\pi \sum_{j=0}^{N_\theta} \hat{u}_{jq} q^2 \int_a^b \sigma(r) \phi_j(r) \phi_p(r) dr.
\end{eqnarray}


Let us define the following matrices:
\begin{eqnarray}
({\bf M}_1)_{jp} & = & \int_a^b r \sigma(r) \frac{d}{dr} \phi_j(r) \phi_p(r) \;dr  \\
({\bf M}_2)_{jp} & = & \int_a^b r^2 \sigma(r) \frac{d}{dr} \phi_j(r) \frac{d}{dr} \phi_p(r) \;dr  \\
({\bf M}_3)_{jp} & = & \int_a^b \sigma(r) \phi_j(r) \phi_p(r) \; dr,
\end{eqnarray}
where $j, p = 0, \ldots, N_\theta$.

$T_4$ becomes:

\begin{eqnarray}
T_4 &=& \int_a^b \int_0^{2\pi} \phi_p(r) e^{iq\theta} r^2 f(r,\theta) d\theta dr \\
    &=& \int_a^b r^2\phi_p(r) \int_0^{2\pi} f(r,\theta) e^{iq\theta} d\theta dr \\
\end{eqnarray}

For given $r$, the Discrete Fourier Transform for $f(r, \theta)$ is defined by
\begin{equation}
f(r, \theta_\tau) = \frac{1}{2\pi} \sum_{k=-N_\theta/2+1}^{N_\theta/2} e^{ik\theta_\tau} \widehat{f(r)}_k
\end{equation}

where
\begin{equation}
\widehat{f(r)}_k = \frac{2\pi}{N_\theta} \sum_{j=1}^{N_\theta} e^{-ik\theta_j} f(r, \theta_j)
\end{equation}
with $ k \in \{-\frac{N_\theta}{2}+1, \cdots, \frac{N_\theta}{2}\}$ and $\theta_j \in \{ \frac{2\pi}{N_\theta}, \cdots, 2\pi  \}$.

Then:
\begin{center}
\begin{eqnarray}
T_4&=& \int_a^b r^2\phi_p(r) \int_0^{2\pi} \frac{1}{2\pi} \sum_{k=-N_\theta/2+1}^{N_\theta/2} e^{ik\theta} \widehat{f(r)}_k e^{iq\theta} d\theta dr \\
%&=& \int_a^b r^2\phi_p(r) \sum_{k=-N_\theta/2+1}^{N_\theta/2} \widehat{f(r)}_k \frac{1}{2\pi} \int_0^{2\pi}  e^{ik\theta} e^{iq\theta} d\theta dr \\
%&=& \int_a^b r^2\phi_p(r) \sum_{k=-N_\theta/2+1}^{N_\theta/2} \widehat{f(r)}_k \delta_{k,q} dr\\
&=& \int_a^b r^2\phi_p(r) \widehat{f(r)}_q dr.
%&=& \sum_\sigma w_\sigma r_\sigma^2\phi_p(r_\sigma) \widehat{f(r_\sigma)}_q.
\end{eqnarray}
\end{center}


\begin{eqnarray}
 2\pi \sum_{j=0}^{N_\theta} \hat{u}_{jq} {M_1}_{jp}
+ 2\pi \sum_{j=0}^{N_\theta} \hat{u}_{jq} {M_2}_{jp} + 2\pi
\sum_{j=0}^{N_\theta} \hat{u}_{jq} q^2 {M_3}_{jp} = \int_a^b
r^2\phi_p(r) \widehat{f(r)}_q dr \\
+ b^2\sigma(b) \phi_p(b) \int_0^{2\pi}e^{iq\theta} {\mathcal
G}_N(\theta) d\theta - a^2\sigma(a) \phi_p(a)
\int_0^{2\pi}e^{iq\theta} \frac{\partial}{\partial r}u(a,\theta)
d\theta
\end{eqnarray}
where $j, p = 0, \ldots, N_r$.

We apply the Fourier transform to the integral term with
${\mathcal G}_N$ and the same idea as one-dimensional case to each
term about boundary conditions.


\section{Experiment Results}
%\{ {\it  sp1d\_ch3\_1.tex} \}

\subsection {H/P Convergence Test for Two-dimensional Solution}

In this section we present the result of convergence in both $h$ refinement and $p$ refinement with the following steady-state Poisson differential equation:
\begin{equation}
\label{pois_sin}
-r^2 \frac{\partial}{\partial r} (\sigma \frac{\partial}{\partial r}u) - r \sigma \frac{\partial}{\partial r}u - \sigma \frac{\partial^2}{\partial \theta^2}u = r \cos \theta \left[ -4 r \pi \cos S_r + \{4\pi^2 + 1\} \sin S_r \right],
\end{equation}
for all $r \in [1, 2], \theta \in [0, 2\pi]$ with $\sigma(r) = r$. The analytic solution is known as 
\begin{equation}
u(r, \theta) = \sin S_r \cos \theta,
\end{equation}
where $S_r = 2\pi (r-1) - \pi$.

The numerical and exact solutions by the solver we developed is shown in figure (\ref{sinsol}).

\begin{figure}[h]
    \begin{center}
    \epsfig{file = figs_dn/sinDN_e10_t8.eps, width = 5cm} %Error: 4.7740e-15
    \epsfig{file = figs_dn/sinDN_e10_t8_1.eps, width = 5cm}
    \caption{\label{sinsol}Numerical and exact solution of equation (\ref{pois_sin}) with polynomial order $P=10$, $10$ equidistance elements. This gives the error 4.7740e-15 to the exact solution}
    \end{center}
\end{figure}

\begin{figure}[h]
    \begin{center}
    \epsfig{file = figs_dn/sinDNhconv.eps, width = 8.3cm}
    \epsfig{file = figs_dn/sinDNpconv.eps, width = 8.3cm}
    \caption{\label{sinDNconv}
(Left)Convergence with respect to discrete $L^{\infty}$ norm as a function of size of elements. This test is performed using the h-type extension with fixed polynomial order 5 and 6 respectively. Error on the Log-Log axis is demonstrating the algebraic convergence of the h-type extension.
(Right)Convergence w.r.t. $L^{\infty}$ norm as a function of size of polynomial order in semi-Log plot. It shows the exponential convergence of p-type extension for smooth solution. Two tests are performed for p-type extension with element length $0.2$ and $0.1$.
}
\end{center}
\end{figure}


\begin{table}[h]
\centering \caption{\label{hconv2t} This table shows the convergence of h-type (left) and p-type (right) resolution control done above Figure (\ref{sinDNconv}). We can see the slopes of each order $P$ is $P+1$ }
\begin{tabular}{|c|c|c|} \hline
    Polynomial order&Error($L^{\infty}$)&Slope   \\ \hline \hline
    5&$9.3603e-013$ &$5.9406$ \\ \hline
    6&$8.6542e-014$ &$6.9402$ \\ \hline
\end{tabular}
\hspace{.5in}
\begin{tabular}{|c|c|} \hline
    Element Size&Error($L^{\infty}$)\\ \hline
    0.2&$1.9385e-012$  \\ \hline
    0.1&$3.4611e-013$  \\ \hline
\end{tabular}

\end{table}

%\{ {\it  sp1d\_ch3\_2.tex} \}

\subsection {High order Polynomial Solution and Its Convergence}

In this section we construct a polynomial $P_n$ of order $n$ defined on $[0,1]$, which satisfies the following.
\begin{eqnarray}
\label{pois_scrv}
    P_n(0) = 0, &P_n(1) = 1 \\
    \frac{d^k}{dx^k}P_n(0) = 0, &\frac{d^k}{dx^k}P_n(1) = 0
\end{eqnarray}
for all $k = 1, \cdots, n-2$. \\
Then for each $n$, we obtain a polynomial $P_n$ by solving a system of linear equations having unique solution which determines the set of coefficients of $P_n$. We will apply the spectral polynomial solver to approximate the second derivative $Q_{n-2}$ of $P_n$.

\begin{figure}[h]
    \begin{center}
    \epsfig{file = figs_dn/ScrvH5_A.eps, width = 5cm}
    \epsfig{file = figs_dn/ScrvH5_N.eps, width = 5cm}
    \caption{\label{scrvsol}Example of curve that satisfies conditions (\ref{pois_scrv}) with polynomial order $n=7, 9, 12, 14, 15$. The error is $2.9239e-8$}
    \end{center}
\end{figure}

We will investigate the convergences by the h-type, p-type extension of trial functions below.

\begin{problem}
Consider the following differential equation for $u(x)$ such that
\begin{equation}
\label{poi_poly}
-r^2 \frac{\partial}{\partial r} (\sigma \frac{\partial}{\partial r}u) - r \sigma \frac{\partial}{\partial r}u - \sigma \frac{\partial^2}{\partial \theta^2}u = e^{-(r-1)^2}\{P_n(r) + (2r^2(r-1)-r)P_n'(r) - r^2 P_n''(r)\} \cos \theta,
\end{equation}
for all $r$ in $[0.1, 1.1]$ with $\sigma(r) = e^{-(r-1)^2}$. Then we have the exact solution as follows.
\begin{equation}
u(r, \theta) = P_n(r) \cos \theta
\end{equation}
where $r$ in $[0.1, 1.1]$ and $\theta$ in $[0, 2\pi]$.
\end{problem}

Note that the accuracy of this interpolation satisfying equations (\ref{pois_scrv}) is dependent on the stability of matrix that defines the coefficients of interpolants. We used the Legendre basis function which is known to be more stable than monomials. In spite of this, there exists interpolation error close to $e^{-13}$. This cause the same amount of convergence error in p-type extension mode shown in right of Figure (\ref{ScrvconvDN}) and Table (\ref{hconv2t}).

\begin{figure}[h]
\begin{center}
\epsfig{file = figs_dn/ScrvHconv.eps, width = 8.3cm}
\epsfig{file = figs_dn/ScrvPconv.eps, width = 8.3cm}
\caption{\label{ScrvconvDN}
(Left)Convergence with respect to discrete $L^{\infty}$ norm as a function of size of elements. This test is performed using the h-type extension with fixed polynomial order 3, 4, and 5 respectively. Error on the Log-Log axis is demonstrating the algebraic convergence of the h-type extension.
(Right)Convergence w.r.t. $L^{\infty}$ norm as a function of size of polynomial order in semi-Log plot. It shows the exponential convergence of p-type extension for smooth solution. Two tests are performed for p-type extension with element length $0.2$ and $0.1$.
}
\end{center}
\end{figure}

\begin{table}[h]
\centering \caption{\label{hconv2t} This table shows the convergence of h-type resolution control done above Figure (\ref{ScrvconvDN}). We can see the slopes of each order $P$ is $P+1$ }
\begin{tabular}{|c|c|c|} \hline
    Polynomial order&Error($L^{\infty}$)&Slope   \\ \hline \hline
    5&$8.5305e-012$ &$5.7556$ \\ \hline
    6&$4.7180e-012$ &$6.8332$ \\ \hline
\end{tabular}
\hspace{.5in}
\begin{tabular}{|c|c|} \hline
    {Element Size}&Error($L^{\infty}$) \\ \hline \hline
    0.2&$9.0616e-13$  \\ \hline
    0.1&$9.4747e-13$ \\ \hline
\end{tabular}
\end{table}


\section{Conclusion}
%\clearpage
%\{ {\it  sp1d\_ch4\_1.tex} \}

Throughout this project, we looked into the theory of Spectral Polynomial and Fourier Method and its solvability. We also investigated the feature of convergence in the viewpoint of h/p convergence property in comparison to classical finite element method. By using Galerkin method, we could incorporate the weak solution and get the problem to be changed to system of linear equations which we can solve it by computer.

By implementing the Spectral element solver for one dimensional Poisson equation having Dirichlet and Neumann boundary conditions, we could experiment all the theory with various specific cases of high-order solutions which were hard to get acceptable convergence in a given time and resolution of domain.  

For the future study, we would like deal with problems regarding:

\begin{itemize}
\item
In supplying $\sigma(r,\theta)$ , we assume $\sigma$ to be dependent only of radius variable. But in general case, we can expand $\sigma$ to be an linear combination of Fourier basis. Computing with only one Fourier basis component, we can easily find the shape of matrix to be dependent of wave number. By adding these matrices up, we could build left hand side stiffness matrix.  
\end{itemize}


%------------------------------------------------------------------------------
%   BIBLIOGRAPHY
%\clearpage

%\{ {\it  sp1d\_chbib.tex} \}

\begin{thebibliography}{9}

\bibitem{Karniadarkis}{\bf Spectral/Hp Element Methods for Cfd},
        George Em Karniadakis, Spencer J. Sherwin, \/
        Oxford Univ Press, 1999.

\bibitem{Trefethen}{\bf Spectral Methods in MATLAB},
        Lloyd N. Trefethen, \/
        Society for Industrial and Applied Mathematics, 2001.

\bibitem{Johnson}{\bf Lecture note of Advanced Methods in Scientific
Computing},
        Christopher R. Johnson, \/
        School of Computing, University of Utah, 2002.

\bibitem{Chen}{\bf A direct spectral collocation Poisson solver in polar and cylindrical
coordinates},
        Chen HL. Su YH. and Shizgal BD., \/
        Journal of Computational Physics. 160(2), 453-469 (2000).

\bibitem{Choe}{\bf Solving One-dimensional Forward Problems using Spectral Element
Method},
        S. Choe, \/
        Project report in Computational Engineering and Science Program, University of Utah, 2004.


\end{thebibliography}



%-------------------------------------------------------------------------------
%   APPENDIX
%
%\input{sp15d_chapdx}


\end{document}
