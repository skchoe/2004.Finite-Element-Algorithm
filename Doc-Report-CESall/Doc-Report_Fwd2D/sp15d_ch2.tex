%\{ {\it  sp1d\_ch2.tex} \}

\subsubsection{Poisson Equation in Polar Coordinates and Basis Functions}

We formulate the Generalized Poisson problem on an annulus $[a, b]\times[0, 2\pi]$, $a > 0$ under the periodic solution $u$ as follows:
\begin{eqnarray}\label{genpois}
-\left[\frac{\partial}{\partial r} (\sigma(r,\theta) \frac{\partial}{\partial r}) + \frac{1}{r} (\sigma(r,\theta) \frac{\partial}{\partial r}) + \frac{1}{r^2}\frac{\partial}{\partial \theta} (\sigma(r,\theta)  \frac{\partial}{\partial \theta})\right] u(r, \theta) = f(r, \theta),\\
\mbox{with periodicity of }u, \;\;\; u(r,0) = u(r,2\pi),
\end{eqnarray}
where $r \in [a, b]$ and $\theta \in [0, 2 \pi]$.

The boundary conditions for this domain is given by:
\begin{equation}
u(a,\theta) = {\mathcal G}_D(\theta), \hspace{1in} \frac{\partial}{\partial r} u(b,\theta) = {\mathcal G}_N(\theta),
\end{equation}
where $\theta \in [0,2\pi]$.

\vspace{0.1in}
The representation of approximation of $u$ is guaranteed by Weierstrass theorem:
\begin{equation}\label{truncapp}
u(r,\theta) = \sum_{j=0}^{N_r} \sum_{k=-N_\theta/2+1}^{N_\theta/2} \hat{u}_{jk} \phi_j(r) e^{ik\theta},
\end{equation}
where $r \in [a, b]$ and $\theta \in [0, 2 \pi]$ for the global degree of freedom $N_r$ and $N_{\theta}$ on $\hat{u}_{jk}$'s.

\vspace{0.1in}

The basis functions $\{\phi_j\}_{j=0}^{N_r}$ shown at linear span
(\ref{truncapp}) are defined as modified Jacobi polynomials
defined in \cite{Karniadarkis}.

\vspace{0.1in}
As a review of discrete Fourier transform in $N$-point grid described in \cite{Trefethen}, the formula for the discrete Fourier transform for $\{v_j\}$ is
\begin{equation}
\hat{v}_k = h \sum_{j=1}^{N} e^{-ikx_j}v_j, \;\; k = -\frac{N}{2}+1, \ldots , \frac{N}{2},
\end{equation}
where $x_j = j\frac{2\pi}{N}$ and the inverse discrete Fourier transform for $\{\hat{v}_k\}$ is given by
\begin{equation}
v_j = \frac{1}{2\pi}\sum_{k = -N/2+1}^{Nr/2}e^{ikx_j}\hat{v}_k,\;\; j = 1, \ldots, N.
\end{equation}


\subsubsection{Formulation of Spectral Polynomial and Fourier Methods}

In this project, we assume the conductivity term $\sigma$ in
equation (\ref{genpois}) to be only dependent on variables showing
the radius domain as we multiply $r^2$ in each side of equation
(\ref{genpois}). Then the Poisson equation in polar coordinate is
as follows:
\begin{eqnarray}
-\left[r^2 \frac{\partial}{\partial r} (\sigma(r) \frac{\partial}{\partial r}) + r \sigma(r) \frac{\partial}{\partial r} + \sigma(r) \frac{\partial^2}{\partial \theta^2}\right] u(r, \theta) = r^2 f(r, \theta).
\end{eqnarray}

\vspace{0.1in}

We utilize the Galerkin method, with test functions of the form:
\begin{equation}
\phi_p(r) e^{iq\theta}, \hspace{.5in} p=0,\ldots,N_r, \;\; q = -\frac{N_\theta}{2}+1,\ldots,\frac{N_\theta}{2}.
\end{equation}

\vspace{0.1in}

The weak form of the equation is as follows:
\begin{equation}\label{galeqn}
-\langle  r^2 \frac{\partial}{\partial r} (\sigma \frac{\partial}{\partial r}u) + r \sigma \frac{\partial}{\partial r}u + \sigma \frac{\partial^2}{\partial \theta^2}u, \phi_p e^{iq\theta} \rangle = \langle r^2 f, \phi_p e^{iq\theta} \rangle.
\end{equation}

\vspace{0.1in} Define $T_i, i = 1,\ldots,4$ as follows
\begin{eqnarray}
T_1 &=& \int_{0}^{2\pi} \int_{a}^{b} \phi_p e^{iq\theta} r^2 \frac{\partial}{\partial r}\left[\sigma(r) \frac{\partial}{\partial r}u(r, \theta)\right] dr d\theta,\\
T_2 &=& \int_{0}^{2\pi} \int_{a}^{b} \phi_p e^{iq\theta} r \sigma(r) \frac{\partial}{\partial r}u(r, \theta) dr d\theta,\\
T_3 &=& \int_{0}^{2\pi} \int_{a}^{b} \phi_p e^{iq\theta} \sigma(r) \frac{\partial^2}{\partial \theta^2}u(r, \theta) dr d\theta,\\
\mbox{and} \hspace{.5in}T_4 &=& \int_{0}^{2\pi} \int_{a}^{b} \phi_p e^{iq\theta} r^2 f(r, \theta) dr d\theta
\end{eqnarray}

Then equation (\ref{galeqn}) becomes:
\begin{equation}\label{smpeqn}
- T_1 - T_2 - T_3 = T_4.
\end{equation}

We obtain boundary terms by integrating by parts on $T_1$ as
follows:
\begin{eqnarray}
T_1 &=& \int_0^{2\pi}e^{iq\theta} \left[ r^2\sigma(r)\frac{\partial}{\partial r}u(r,\theta)\phi_p(r)\right]_a^b d\theta \\
&-&2 \int_0^{2\pi}\int_a^b e^{iq\theta} r \sigma(r)\frac{\partial}{\partial r}u(r,\theta)\phi_p(r)drd\theta \\
&-& \int_0^{2\pi}\int_a^b e^{iq\theta} r^2\sigma(r)\frac{\partial}{\partial r}u(r,\theta) \frac{d}{dr}\phi_p(r)drd\theta.
\end{eqnarray}

Then the right hand side of (\ref{smpeqn}) becomes
\begin{eqnarray}\label{rhseqn1}
- T_1 - T_2 - T_3 &=& - \int_0^{2\pi}e^{iq\theta} \left[ r^2\sigma(r)\frac{\partial}{\partial r}u(r,\theta)\phi_p(r)\right]_a^b d\theta \\
&+& \int_0^{2\pi}\int_a^b e^{iq\theta} r \sigma(r) \frac{\partial}{\partial r}u(r,\theta)\phi_p(r)drd\theta \\
&+& \int_0^{2\pi}\int_a^b e^{iq\theta} r^2 \sigma(r) \frac{\partial}{\partial r}u(r,\theta) \frac{d}{dr}\phi_p(r)drd\theta \\
&-& \int_0^{2\pi}\int_a^b e^{iq\theta} \sigma(r) \frac{\partial^2}{\partial \theta^2}u(r, \theta) \phi_p(r) dr d\theta
\end{eqnarray}

Using (\ref{truncapp}) and the orthogonal properties of
$\{e^{ik\theta}\}$, $k =
-\frac{N_\theta}{2}+1,\ldots,\frac{N_\theta}{2}$, we can simplify
(\ref{rhseqn1}) to obtain:

\begin{eqnarray}\label{rhseqn2}
- T_1 - T_2 - T_3 &=& - \int_0^{2\pi}e^{iq\theta} \left[ r^2\sigma(r)\frac{\partial}{\partial r}u(r,\theta)\phi_p(r)\right]_a^b d\theta \\
&+& 2\pi \sum_{j=0}^{N_\theta} \hat{u}_{jq} \int_a^b r \sigma(r) \frac{d}{dr} \phi_j(r) \phi_p(r) dr \\
&+& 2\pi \sum_{j=0}^{N_\theta} \hat{u}_{jq} \int_a^b r^2 \sigma(r) \frac{d}{dr} \phi_j(r) \frac{d}{dr}\phi_p(r) dr \\
&+& 2\pi \sum_{j=0}^{N_\theta} \hat{u}_{jq} q^2 \int_a^b \sigma(r) \phi_j(r) \phi_p(r) dr.
\end{eqnarray}


Let us define the following matrices:
\begin{eqnarray}
({\bf M}_1)_{jp} & = & \int_a^b r \sigma(r) \frac{d}{dr} \phi_j(r) \phi_p(r) \;dr  \\
({\bf M}_2)_{jp} & = & \int_a^b r^2 \sigma(r) \frac{d}{dr} \phi_j(r) \frac{d}{dr} \phi_p(r) \;dr  \\
({\bf M}_3)_{jp} & = & \int_a^b \sigma(r) \phi_j(r) \phi_p(r) \; dr,
\end{eqnarray}
where $j, p = 0, \ldots, N_\theta$.

$T_4$ becomes:

\begin{eqnarray}
T_4 &=& \int_a^b \int_0^{2\pi} \phi_p(r) e^{iq\theta} r^2 f(r,\theta) d\theta dr \\
    &=& \int_a^b r^2\phi_p(r) \int_0^{2\pi} f(r,\theta) e^{iq\theta} d\theta dr \\
\end{eqnarray}

For given $r$, the Discrete Fourier Transform for $f(r, \theta)$ is defined by
\begin{equation}
f(r, \theta_\tau) = \frac{1}{2\pi} \sum_{k=-N_\theta/2+1}^{N_\theta/2} e^{ik\theta_\tau} \widehat{f(r)}_k
\end{equation}

where
\begin{equation}
\widehat{f(r)}_k = \frac{2\pi}{N_\theta} \sum_{j=1}^{N_\theta} e^{-ik\theta_j} f(r, \theta_j)
\end{equation}
with $ k \in \{-\frac{N_\theta}{2}+1, \cdots, \frac{N_\theta}{2}\}$ and $\theta_j \in \{ \frac{2\pi}{N_\theta}, \cdots, 2\pi  \}$.

Then:
\begin{center}
\begin{eqnarray}
T_4&=& \int_a^b r^2\phi_p(r) \int_0^{2\pi} \frac{1}{2\pi} \sum_{k=-N_\theta/2+1}^{N_\theta/2} e^{ik\theta} \widehat{f(r)}_k e^{iq\theta} d\theta dr \\
%&=& \int_a^b r^2\phi_p(r) \sum_{k=-N_\theta/2+1}^{N_\theta/2} \widehat{f(r)}_k \frac{1}{2\pi} \int_0^{2\pi}  e^{ik\theta} e^{iq\theta} d\theta dr \\
%&=& \int_a^b r^2\phi_p(r) \sum_{k=-N_\theta/2+1}^{N_\theta/2} \widehat{f(r)}_k \delta_{k,q} dr\\
&=& \int_a^b r^2\phi_p(r) \widehat{f(r)}_q dr.
%&=& \sum_\sigma w_\sigma r_\sigma^2\phi_p(r_\sigma) \widehat{f(r_\sigma)}_q.
\end{eqnarray}
\end{center}


\begin{eqnarray}
 2\pi \sum_{j=0}^{N_\theta} \hat{u}_{jq} {M_1}_{jp}
+ 2\pi \sum_{j=0}^{N_\theta} \hat{u}_{jq} {M_2}_{jp} + 2\pi
\sum_{j=0}^{N_\theta} \hat{u}_{jq} q^2 {M_3}_{jp} = \int_a^b
r^2\phi_p(r) \widehat{f(r)}_q dr \\
+ b^2\sigma(b) \phi_p(b) \int_0^{2\pi}e^{iq\theta} {\mathcal
G}_N(\theta) d\theta - a^2\sigma(a) \phi_p(a)
\int_0^{2\pi}e^{iq\theta} \frac{\partial}{\partial r}u(a,\theta)
d\theta
\end{eqnarray}
where $j, p = 0, \ldots, N_r$.

We apply the Fourier transform to the integral term with
${\mathcal G}_N$ and the same idea as one-dimensional case to each
term about boundary conditions.
