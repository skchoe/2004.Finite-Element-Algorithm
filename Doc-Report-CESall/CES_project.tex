\documentclass[11pt, notitlepage,  letterpaper]{article}


%% Horizontal Lengths - Max 6.5
\setlength{\footskip}{0.5in}
\setlength{\hoffset}{0in}
\setlength{\oddsidemargin}{-.2in}
\setlength{\evensidemargin}{-.2in}
%\setlength{\textwidth}{6.9in}
\setlength{\textwidth}{6.6in}

%% Vertical Lengths - Max 9.0
\setlength{\headheight}{0in} \setlength{\topskip}{0in}
\setlength{\voffset}{0in} \setlength{\topmargin}{0.5in}
\setlength{\textheight}{8.5in}

%\usepackage{fancyheadings}

\usepackage{url}
\usepackage[]{epsfig,amsmath}[]
%\usepackage{doublespace}

%\pagestyle{headings}
%\pagestyle{fancy}
\pagenumbering{arabic}

\renewcommand{\baselinestretch}{1.15}

%%%%%%%%%%%%%%%%%%%%%%%%%%%%%%%%%%%%%%%%%%%%%%%%%%%%%%%%%%%%%%%%%%%%%%
% New definitions and commands
\newtheorem{define}{Definition}[section]
\newtheorem{theorem}[define]{Theorem}
\newtheorem{question}[define]{Question}
\newtheorem{problem}[define]{Problem}


%**************************************

\begin{document}
%\title{{\normalsize Fall 2003, Semester research;}\\
\title{
{Solving the Poisson Partial Differential Equation}\\
{using Spectral Polynomial Methods }
}

 \vfill
\author{Seungkeol\ Choe \thanks{Computational Engineering and Science Program, University of Utah}}

\renewcommand{\today}{April 18th, 2004}

\maketitle

\begin{abstract}
In this report, we present a spectral polynomial method for
solving the Poisson equation with Dirichlet and Neumann boundary
conditions respectively on a one dimensional compact interval. We
can control the number, size of elements, and order of
approximating polynomials to obtain accurate solutions with faster
convergence than the classical finite element method. We present a
spectral Fourier method for solving a Poisson equation with the
same boundary conditions on an two dimensional annulus domain. By
using the tensor product between one dimensional spectral bases
and Fourier bases, we apply the high-order method to radius
direction and the fast Fourier transform to angular direction.
\end{abstract}

%\noinden {\bf Keywords: Spectral Element Methods, Poisson Equation}
\clearpage

\tableofcontents

%-------------------------------------------------------------------------------
\clearpage

%\begin{spacing}{2}

\section{Introduction}

%\{ {\it  sp1d\_ch1.tex} \}

The spectral method is a numerical scheme for approximating the
solution of partial differential equations. It has developed
rapidly in the past three decades and has been applied to
numerical simulation problems in many fields.

One reason for its broad and fast acceptance is its use of various
systems of infinitely differentiable basis functions as trial
functions. By choosing an appropriate orthogonal system based on
domain of orthogonality, we can apply the method to periodic and
non-periodic problems, as well as problems defined on various
domains, such as compact domain, half/all intervals.

Another advantage of the spectral method is its high accuracy. In
particular, the spectral polynomial method allows control of both
the resolution of element size and the order of approximation.
This results in exponential convergence, which is a marked
improvement over classical finite difference and other purely
element methods.

For solving the problem, we discretize the domain and obtain the
following approximation of infinite series. A function $u$ is
represented via a truncated series expansion as follows:
\begin{equation}
\label{solapprx}
u \approx u_N = \sum_{n=0}^{N} \hat u_n \phi_n,
\end{equation}
where $\phi_n$ are the basis functions. In general the Chebyshev
polynomials $T_n$ or the Legendre polynomials $L_n$ or another
member of the class of Jacobi polynomials $P_n^{\alpha, \beta}$
are employed as basis functions.

In choosing proper basis function, we apply the following 3
characteristics to the candidate basis.
\begin{itemize}
\item {\bf Numerical Efficiency} After discretization,
the choice of basis affects complexity of mass matrix. Moreover we
need consider the efficiency in solving the system of linear
equations. For example, using the monomials $\{x^k\}_{k=0}^{N}$
result in the matrix whose non-zero component are the half. But
its inverse matrix is full and the cost of inverting is dominant.
In contrast, the Legendre polynomial basis composes diagonal mass
matrix and its inverse matrix can be calculated very efficiently.
\item {\bf Conditioning} If a matrix system is ill-conditioned
the round-off error in the matrix system can lead to large errors
in the solution. Furthermore the number iteration required in
inverting the matrix using iterative solver can be increase by the
condition number. The condition number of monomials and Lagrange
polynomial are close to $10^p$, for the polynomial order $p$. But
Legendre polynomial has condition number $2P+1$. This condition
also affects the degree of linear independence of each basis
function.
\end{itemize}

This project seeks to study the fundamental theory of spectral
method, problems, solvability, and to obtain its constructive
procedure to apply the method to various application problems. By
checking solutions obtained via the spectral method against exact
solutions, we can validate the method and see how much we can save
the effort on discretisation of domain to achieve the same degree
of accuracy in comparison to classical element methods.

In chapter 2, I investigate spectral method solutions for the
forward Poisson problem with Dirichlet and Neumann boundary
condition on each end of one dimensional interval domain. In
chapter 3, I investigate the spectral polynomial method and
Fourier method for solving the forward Poisson problem with
Dirichlet and Neumann boundary condition on inner and outer
boundary circles respectively. Our approach utilizes spectral
polynomial element and Fourier method in tensor product form. As
an conclusion we present the result of numerical solution and its
convergence by h/p adaptive control.

\subsection{Poisson Equation}

Science and engineering disciplines are generally interested in
systems of continuous quantities and relations. This project
focuses on solutions of the Poisson equation, which appears in
various field such as electrostatics, magnetics, heat flow,
elastic membranes, torsion, and fluid flow.

For example, electrostatics, the governing equation appears as
Gauss's law in differential form \cite{Johnson}:
\begin{equation}
\nabla \cdot \mathbf{E} = 4 \pi \rho
\end{equation}
which indicates that the charge within a closed spherical surface
is related to the electric field \textbf{E} normal to surface
element where $\rho$ is a charge density.

Since it is known in electrostatics that the electric field \textbf{E} is conservative, \textbf{E} is a form of gradient of a scalar potential $\Phi$,
\begin{equation}
\mathbf{E} = - \nabla\Phi.
\end{equation}
With these two relationships, we obtain a Poisson equation
\begin{equation}
\nabla^2\Phi = -4\pi\rho.
\end{equation}


In this report, the one-dimensional Poisson equation is defined as

\begin{equation}
\label{poisson1} L(u) \equiv  \frac{d^2}{dx^2} u + f = 0.
\end{equation}
where $u$ and $f$ are defined on $\Omega$.

In pointwise viewpoint, the one dimensional Poisson equation
(\ref{poisson1}) is written as
\begin{equation}
\label{poisson2} L(u)(x) \equiv \frac{d^2}{dx^2} u(x) + f(x) = 0,
\end{equation}
for all $x$ in $(a, b)$.

\subsection{Method of Weighted Residuals}
According to the Weierstrass approximation theorem, for any given
real valued continuous solution $u$ on a compact interval $[a, b]$
we can obtain real polynomial function $p$ of certain degree such
that $p$ uniformly approximates $u$. Although the result of
convergence at each point is within a predefined error bound, this
does not satisfy the requirement that we need to acquire an
accurate solution on a specific situation. By imposing certain
restrictions, we can obtain a formulation that satisfies the
requirement.

To describe this, we set a general linear differential equation on $\Omega$.
\begin{equation}
\label{pde1} L(u) = 0.
\end{equation}
with appropriate initial and boundary conditions. Under certain
restriction,  we assume that the true solution $u(x)$ can be
approximated by a finite series expansion of the form:
\begin{equation}
\label{sol1} u^{\delta}(x) = \sum_{i=0}^{N_{dof}-1} \hat u_i
\Phi_i(x),
\end{equation}
where $\Phi_i(x)$ are polynomial trial functions and $\hat u_i$
are $N_{dof}$ unknown coefficients. We assume the following:
\begin{eqnarray}
    \hat u_0  &=& \mathcal{G}_{D} \qquad \mbox{: Dirichlet Boundary Value}, \\
    \Phi_0(a) &=& 1,  \Phi_{N_{dof}-1}(b) = 1 \qquad \mbox{where $a, b$ are the boundary of domain $\Omega$}\\
\end{eqnarray}
We then define a non-zero residual $R$ by:
\begin{equation}
R(u^{\delta}) = L(u^{\delta}).
\end{equation}

Define a set of functions, $H^1(\Omega)$  and a norm
$||\cdot||_{H^1(\Omega)}$ on the space as follows:
\begin{eqnarray}
H^1(\Omega) &=& \{v \in L^2(\Omega) : \frac{d}{dx}v \in L^2(\Omega)\}, \\
||v ||_{H^1(\Omega)} &=& \left[ \int_{\Omega}v(x)^2 + \frac{d}{dx}v(x)^2 dx \right]^{\frac{1}{2}}, \quad v \in H^1(\Omega).
\end{eqnarray}

We define an inner-product $\langle \cdot, \cdot \rangle$ over
$H^1(\Omega)$  as follows:
\begin{equation}
\label{functional}
\langle u, v \rangle = \int_{\Omega} u(x) \cdot v(x) dx,
\end{equation}

This method is restricted to test functions, $v(x)$ that satisfy:
\begin{equation}
\langle v, R \rangle = 0.
\end{equation}

For example, in the collocation method, the $j^{th}$ test function
is the Dirac delta function which evaluates to a collocation point
$x = x_j$. Then we have
\begin{equation}
0 = \langle \delta_j, R \rangle = \int_{\Omega} R(u^{\delta})(x)\delta_j(x)dx = R(u^{\delta})(x_j) = L(u^{\delta})(x_j).
\end{equation}

Other possible test functions are examined in \cite{Karniadarkis}.


\clearpage
\section{Spectral Polynomial Elements Method on an Interval}

\subsection{Mathematical Formulations}
\section {Spectral Polynomial Methods in 1 Dimensional Space}

Our problem is the Poisson equation
\begin{equation}
\label{poisson1} L(u) \equiv  \nabla^2 u - f = 0,
\end{equation}
for all $x \in \Omega$. In one dimensional case, \ref{poisson1} is
written as
\begin{equation}
\label{poisson2} L(u)(x) \equiv \frac{d^2}{dx^2} u(x) - f(x) = 0,
\end{equation}
for all $x$ in $[a, b]$.

We solve this equation in weak sense. That is to say, we define a
functional $\left( \cdot \right) : C^0 \rightarrow \Re$ such that
\begin{equation}
\left(\nabla^2 u, \nu \right) = \int_{\Omega} \frac{d^2}{dx^2}
u(x) \cdot \nu(x) d\mu(x),
\end{equation}
for each $\nu$ in $C^0$, where $C^0$ is a set of continuous
functions. Then we find the solution u by solving the equation as
follows:

\begin{equation} \label{weakabs}
\left(\nabla^2 u, \nu \right) = \left(f, \nu\right),
\end{equation}
for each $\nu$ in $C^0$.

%--------------------------------------------------------------------------
\subsection {Basis Functions}

The spectral approximation of solution $u$ is generally
represented as
\begin{equation}\label{genrep}
u(x) = \sum_{i=0}^{N_{dof-1}} \hat u_i(x)\Psi_i(x)
\end{equation}
on $[a, b]$. To construct this the global basis functions
$\{\Psi_i(x)\}_{i=0}^{N_{dof-1}}$, each $\Psi$ is represented by
the linear combination of local basis functions $\psi_i$ on each
element in $[a, b]$, say $\Omega^e$.

We define a basis functions ${\psi_i}$ on $\Omega^{st}$ to be a
real valued function with the Legendre polynomial $\{P_i^{1,1}\}$
as follows:
\begin{equation}\label{locbasis}
  \psi_i(\xi) =\large \{
  \begin{array}{ll}
  \frac{1-\xi}{2}, & i=0 \\
  \frac{1+\xi}{2}, & i=1 \\
  \left(\frac{1-\xi}{2}\right)\left(\frac{1+\xi}{2}\right)P_{i-2}^{1,1}(\xi),
  &i\ge 2
  \end{array}
\end{equation}
for all $\xi$ in $[-1, 1]$.

Then on a single standard element $\Omega^{st}$, the approximation
$u(\xi)$ is represented as
\begin{equation}\label{locrep}
u(\xi) = \sum_{i=0}^{N^e} \hat u_{i}^{e}\psi_i(\xi),
\end{equation}
for $\xi$ in $\Omega^{st}$.

%--------------------------------------------------------------------------
\subsection {Spectral Polynomial Method in A Element}


We apply the basis representation \ref{locrep} to weak formulation
\ref{weakabs} with the same test function $\{\psi_q\}$, then we
obtain the following:
\begin{equation}\label{locmat}
 \sum_{p=0}^{N^e} \hat u_p^e \left(\nabla^2 \psi_p, \psi_q \right)
= \left(\nabla^2 \sum_{p=0}^{N^e} u_p^e \psi_p, \psi_q\right) =
\left( f, \psi_q \right)
\end{equation}
for $q = 0, \cdots, N^e$.

With this relation we can setup a system of linear equations for
the coefficient $\{\hat u_p^e\}_{p=0}^{N^e}$ with $N^e+1 \times
N^e+1$ matrix $\mathbf{L}_{N^e}$ defined as follows:
\begin{equation}\label{loceqn}
    \mathbf{L}_{N^e} \cdot \mathbf{\hat u} = \mathbf{f},
\end{equation}
where
\begin{eqnarray}\label{locdefs}
\mathbf{L}_{N^e}(p,q) &=
\int_{\Omega^e}\frac{d^2}{d\xi^2}\psi_p(\xi)\psi_q(\xi) d\xi, \\
\mathbf{\hat u} &= \left[\hat u_p\right]_{p=0}^{N^e}, \\
\mathbf{f} &= \left[ \int_{\Omega^e} f(\xi) \psi_q(\xi) d\xi
\right]_{q=0}^{N^e}.
\end{eqnarray}

\subsection {H/P Refinement using Global Assembly}


\subsection{Experiment Results}
In section 1, 2, we present the result of convergence in both h
refinement and p refinement with the following steady-state
Poisson differential equation:
\begin{equation*}
    \frac{d^2}{dx^2} u(x) = \sin(\pi x),
\end{equation*}
for all x in $[0, 1]$.


\subsection {H-Convergence of 1-D Spectral Method}

This test is to validate the relation between size of element and
the accuracy of approximation. We apply equidistance element and
investigate the movement of error scale. As shown in Figure
\ref{sinDDhconv} and \ref{sinDNhconv}, the smaller are the
elements, the more exact is the solution. Moreover by testing with
different order of basis, we also could see the fact that the
higher are orders, the faster do they converge.


\begin{itemize}

\item Dirichlet-Dirichlet Case
\begin{figure}[h]
\begin{center}
\epsfig{file = figs_dd/sinDDhconv.eps, %
        height = 9cm}
\caption{\label{sinDDhconv}Graph showing change of errors by the
increase of the number of elements: Dirichlet-Dirichlet}
\end{center}
\end{figure}

\begin{table}[h]
\centering \caption{\label{hconv1t} Specification of
                              Figure\ref{sinDDhconv} and their errors}
\begin{tabular}{|c|c|c|} \hline
Polynomial order&Error&Slope   \\ \hline \hline
    3&$1.1940e-012$ &$3.9946$ \\ \hline
    4&$2.2204e-015$ &$4.9875$ \\ \hline
    5&$2.4147e-015$ &$5.9839$ \\ \hline
\end{tabular}
\end{table}

\item Dirichlet-Neumann Case


\begin{figure}[h]
\begin{center}
\epsfig{file = figs_dn/sinDNhconv.eps, %
        height = 9cm}
\caption{\label{sinDNhconv}Graph showing change of errors by the
increase of the number of elements: Dirichlet-Dirichlet}
\end{center}
\end{figure}

\begin{table}[h]
\centering \caption{\label{hconv2t} Specification of
                              Figure\ref{sinDNhconv} and their errors}
\begin{tabular}{|c|c|c|} \hline
Polynomial order&Error&Slope   \\ \hline \hline
    3&$1.1620e-012$ &$4.0024$ \\ \hline
    4&$4.6629e-014$ &$4.9877$ \\ \hline
    5&$9.7367e-014$ &$5.9775$ \\ \hline
\end{tabular}
\end{table}


\end{itemize}



\subsection {P-Convergence of 1-D Spectral Method}



\begin{itemize}

\item Dirichlet-Dirichlet Case
\begin{figure}[h]
\begin{center}
\epsfig{file = figs_dd/sinDDpconv.eps, %
        height = 9cm}
\caption{\label{sinDDpconv}Graph showing change of errors by the
increase of the number of elements: Dirichlet-Dirichlet}
\end{center}
\end{figure}

\begin{table}[h]
\centering \caption{\label{pconv1t} Specification of
                              Figure\ref{sinDDpconv} and their errors}
\begin{tabular}{|c|c|} \hline
Element Size&Error   \\ \hline \hline
    0.2&$7.7716e-016$  \\ \hline
    0.1&$1.1796e-016$  \\ \hline
\end{tabular}
\end{table}

\item Dirichlet-Neumann Case


\begin{figure}[h]
\begin{center}
\epsfig{file = figs_dn/sinDNpconv.eps, %
        height = 9cm}
\caption{\label{sinDNpconv}Graph showing change of errors by the
increase of the number of elements: Dirichlet-Neumann}
\end{center}
\end{figure}

\begin{table}[h]
\centering \caption{\label{pconv2t} Specification of
                              Figure\ref{sinDNpconv} and their errors}
\begin{tabular}{|c|c|} \hline
Element Size&Error   \\ \hline \hline
    0.2&$8.3267e-016$  \\ \hline
    0.1&$6.6613e-016$  \\ \hline
\end{tabular}
\end{table}


\end{itemize}

%\{ {\it  sp1d\_ch3\_2.tex} \}

\begin{figure}[h]
    \begin{center}
    \epsfig{file = Doc-Report_Fwd1D/figs_dn/ScrvO_13.eps, width = 5cm}
    \caption{\label{scrvsol1}Example of a curve satisfing conditions (\ref{pois_scrv1}), with polynomial order $n=9$}.
    \end{center}
\end{figure}

\subsubsection {High order Polynomial Solution and Its Convergence}

In this section we construct a polynomial $P_n$ of order $n$ defined on $[0,1]$, which satisfies the following.
\begin{eqnarray}
\label{pois_scrv1}
    P_n(0) = 0, &P_n(1) = 1 \\
    \frac{d^k}{dx^k}P_n(0) = 0, &\frac{d^k}{dx^k}P_n(1) = 0
\end{eqnarray}
for all $k = 1, \cdots, n-2$. \\
For each $n$, we obtain a polynomial $P_n$ by solving a system of
linear equations that determines the coefficients of $P_n$. We
apply the spectral polynomial solver to approximate the second
derivative $Q_{n-2}$ of $P_n$. The numerical and exact solutions
by the solver we developed is shown in figure (\ref{scrvsol1}).



\begin{problem}
Consider the following differential equation for $u(x)$ such that
\begin{equation}
\label{poi_poly1}
    \frac{d^2}{dx^2} u(x) = Q_{n-2},
\end{equation}
for all $x$ in $[0, 1]$ with the boundary condition defined in
equation (\ref{pois_scrv1}). Approximate $u(x)$ using spectral
polynomial method.
\end{problem}

Note that the accuracy of the interpolation satisfying equations
(\ref{pois_scrv1}) is dependent on the stability of the matrix
defining the coefficients of interpolants. We used the Legendre
basis functions because they are known to be more stable than
monomials. Despite this choise, interpolation error is nearly
$e^{-13}$. This results in the same amount of convergence error in
p-type extension mode shown in right of Figure (\ref{ScrvconvDN1})
and Table (\ref{hconv2t1}).

\begin{enumerate}

\item {Convergence h-type extension for equation (\ref{poi_poly1})}
Examining the equation (\ref{pois_sin1}), in Figure
(\ref{ScrvconvDN1}), we observe that the error with respect to
$L^{\infty}$ of the discrete solution to the equation is
exponentially convergent with respect to the size of element. This
verifies the Log-Log scale of relation of theory
(\ref{hrelation}).
\item {Convergence p-type extension for equation (\ref{poi_poly1})}
This semi-Log scale plot also shows the exponential convergence of
p-type extension of trial functions. Note that we approximate the
finite order of the polynomials. So there exists the lowest order
$P_l$ that approximates with trial functions of order $P$ which $P
> P_l$ should shows the same convergence as the case using trial
functions of order $P_l$. In right of Figure (\ref{ScrvconvDN1}),
we observe that the convergence is staying on approximation error
which theoretically should be machine precision.
\end{enumerate}

\begin{figure}[h]
\begin{center}
\epsfig{file = Doc-Report_Fwd1D/figs_dn/ScrvHconv.eps, width = 8.3cm}
\epsfig{file = Doc-Report_Fwd1D/figs_dn/ScrvPconv.eps, width = 8.3cm}
\caption{\label{ScrvconvDN1}
(Left) Convergence with respect to discrete $L^{\infty}$ norm as a function of  element
size. This test is performed using the h-type extension with fixed
polynomial order 3, 4, and 5 respectively. Error on the Log-Log
axis demonstrates the algebraic convergence of the h-type
extension.
(Right) Convergence w.r.t. $L^{\infty}$ norm as a function of size
of polynomial order in semi-Log plot. It shows the exponential
convergence of p-type extension for smooth solution. The two tests
are performed for p-type extension with element lengths of $0.2$
and $0.1$. }
\end{center}
\end{figure}

\begin{table}[h]
\centering \caption{\label{hconv2t1} This table shows the
convergence of h-type resolution control done above Figure
(\ref{ScrvconvDN1}). Note that the slopes of each order $P$ is
$P+1$ }
\begin{tabular}{|c|c|c|} \hline
    Polynomial order&Error($L^{\infty}$)&Slope   \\ \hline \hline
    3&$7.5530e-011$ &$3.9908$ \\ \hline
    4&$4.5619e-013$ &$4.6486$ \\ \hline
    5&$4.1855e-013$ &$5.7218$ \\ \hline
\end{tabular}
\hspace{.5in}
\begin{tabular}{|c|c|} \hline
    &\multicolumn{1}{|c|}{Error}\\
    \raisebox{0.5\baselineskip}%
    {Element Size}&($L^{\infty}$) \\ \hline \hline
    0.2&$3.0431e-013$  \\ \hline
    0.1&$3.1186e-013$ \\ \hline
\end{tabular}
\end{table}


%\clearpage
%\vspace*{3cm}
\clearpage
\section{Spectral Fourier Method on 2-dimensional Annulus}

\subsection{Mathematical Formulations}
%\{ {\it  sp1d\_ch2.tex} \}

\subsubsection{Poisson Equation in Polar Coordinates and Basis Functions}

We formulate the Generalized Poisson problem on an annulus $[a, b]\times[0, 2\pi]$, $a > 0$ under the periodic solution $u$ as follows:
\begin{eqnarray}\label{genpois}
-\left[\frac{\partial}{\partial r} (\sigma(r,\theta) \frac{\partial}{\partial r}) + \frac{1}{r} (\sigma(r,\theta) \frac{\partial}{\partial r}) + \frac{1}{r^2}\frac{\partial}{\partial \theta} (\sigma(r,\theta)  \frac{\partial}{\partial \theta})\right] u(r, \theta) = f(r, \theta),\\
\mbox{with periodicity of }u, \;\;\; u(r,0) = u(r,2\pi),
\end{eqnarray}
where $r \in [a, b]$ and $\theta \in [0, 2 \pi]$.

The boundary conditions for this domain is given by:
\begin{equation}
u(a,\theta) = {\mathcal G}_D(\theta), \hspace{1in} \frac{\partial}{\partial r} u(b,\theta) = {\mathcal G}_N(\theta),
\end{equation}
where $\theta \in [0,2\pi]$.

\vspace{0.1in}
The representation of approximation of $u$ is guaranteed by Weierstrass theorem:
\begin{equation}\label{truncapp}
u(r,\theta) = \sum_{j=0}^{N_r} \sum_{k=-N_\theta/2+1}^{N_\theta/2} \hat{u}_{jk} \phi_j(r) e^{ik\theta},
\end{equation}
where $r \in [a, b]$ and $\theta \in [0, 2 \pi]$ for the global degree of freedom $N_r$ and $N_{\theta}$ on $\hat{u}_{jk}$'s.

\vspace{0.1in}

The basis functions $\{\phi_j\}_{j=0}^{N_r}$ shown at linear span
(\ref{truncapp}) are defined as modified Jacobi polynomials
defined in \cite{Karniadarkis}.

\vspace{0.1in}
As a review of discrete Fourier transform in $N$-point grid described in \cite{Trefethen}, the formula for the discrete Fourier transform for $\{v_j\}$ is
\begin{equation}
\hat{v}_k = h \sum_{j=1}^{N} e^{-ikx_j}v_j, \;\; k = -\frac{N}{2}+1, \ldots , \frac{N}{2},
\end{equation}
where $x_j = j\frac{2\pi}{N}$ and the inverse discrete Fourier transform for $\{\hat{v}_k\}$ is given by
\begin{equation}
v_j = \frac{1}{2\pi}\sum_{k = -N/2+1}^{Nr/2}e^{ikx_j}\hat{v}_k,\;\; j = 1, \ldots, N.
\end{equation}


\subsubsection{Formulation of Spectral Polynomial and Fourier Methods}

In this project, we assume the conductivity term $\sigma$ in
equation (\ref{genpois}) to be only dependent on variables showing
the radius domain as we multiply $r^2$ in each side of equation
(\ref{genpois}). Then the Poisson equation in polar coordinate is
as follows:
\begin{eqnarray}
-\left[r^2 \frac{\partial}{\partial r} (\sigma(r) \frac{\partial}{\partial r}) + r \sigma(r) \frac{\partial}{\partial r} + \sigma(r) \frac{\partial^2}{\partial \theta^2}\right] u(r, \theta) = r^2 f(r, \theta).
\end{eqnarray}

\vspace{0.1in}

We utilize the Galerkin method, with test functions of the form:
\begin{equation}
\phi_p(r) e^{iq\theta}, \hspace{.5in} p=0,\ldots,N_r, \;\; q = -\frac{N_\theta}{2}+1,\ldots,\frac{N_\theta}{2}.
\end{equation}

\vspace{0.1in}

The weak form of the equation is as follows:
\begin{equation}\label{galeqn}
-\langle  r^2 \frac{\partial}{\partial r} (\sigma \frac{\partial}{\partial r}u) + r \sigma \frac{\partial}{\partial r}u + \sigma \frac{\partial^2}{\partial \theta^2}u, \phi_p e^{iq\theta} \rangle = \langle r^2 f, \phi_p e^{iq\theta} \rangle.
\end{equation}

\vspace{0.1in} Define $T_i, i = 1,\ldots,4$ as follows
\begin{eqnarray}
T_1 &=& \int_{0}^{2\pi} \int_{a}^{b} \phi_p e^{iq\theta} r^2 \frac{\partial}{\partial r}\left[\sigma(r) \frac{\partial}{\partial r}u(r, \theta)\right] dr d\theta,\\
T_2 &=& \int_{0}^{2\pi} \int_{a}^{b} \phi_p e^{iq\theta} r \sigma(r) \frac{\partial}{\partial r}u(r, \theta) dr d\theta,\\
T_3 &=& \int_{0}^{2\pi} \int_{a}^{b} \phi_p e^{iq\theta} \sigma(r) \frac{\partial^2}{\partial \theta^2}u(r, \theta) dr d\theta,\\
\mbox{and} \hspace{.5in}T_4 &=& \int_{0}^{2\pi} \int_{a}^{b} \phi_p e^{iq\theta} r^2 f(r, \theta) dr d\theta
\end{eqnarray}

Then equation (\ref{galeqn}) becomes:
\begin{equation}\label{smpeqn}
- T_1 - T_2 - T_3 = T_4.
\end{equation}

We obtain boundary terms by integrating by parts on $T_1$ as
follows:
\begin{eqnarray}
T_1 &=& \int_0^{2\pi}e^{iq\theta} \left[ r^2\sigma(r)\frac{\partial}{\partial r}u(r,\theta)\phi_p(r)\right]_a^b d\theta \\
&-&2 \int_0^{2\pi}\int_a^b e^{iq\theta} r \sigma(r)\frac{\partial}{\partial r}u(r,\theta)\phi_p(r)drd\theta \\
&-& \int_0^{2\pi}\int_a^b e^{iq\theta} r^2\sigma(r)\frac{\partial}{\partial r}u(r,\theta) \frac{d}{dr}\phi_p(r)drd\theta.
\end{eqnarray}

Then the right hand side of (\ref{smpeqn}) becomes
\begin{eqnarray}\label{rhseqn1}
- T_1 - T_2 - T_3 &=& - \int_0^{2\pi}e^{iq\theta} \left[ r^2\sigma(r)\frac{\partial}{\partial r}u(r,\theta)\phi_p(r)\right]_a^b d\theta \\
&+& \int_0^{2\pi}\int_a^b e^{iq\theta} r \sigma(r) \frac{\partial}{\partial r}u(r,\theta)\phi_p(r)drd\theta \\
&+& \int_0^{2\pi}\int_a^b e^{iq\theta} r^2 \sigma(r) \frac{\partial}{\partial r}u(r,\theta) \frac{d}{dr}\phi_p(r)drd\theta \\
&-& \int_0^{2\pi}\int_a^b e^{iq\theta} \sigma(r) \frac{\partial^2}{\partial \theta^2}u(r, \theta) \phi_p(r) dr d\theta
\end{eqnarray}

Using (\ref{truncapp}) and the orthogonal properties of
$\{e^{ik\theta}\}$, $k =
-\frac{N_\theta}{2}+1,\ldots,\frac{N_\theta}{2}$, we can simplify
(\ref{rhseqn1}) to obtain:

\begin{eqnarray}\label{rhseqn2}
- T_1 - T_2 - T_3 &=& - \int_0^{2\pi}e^{iq\theta} \left[ r^2\sigma(r)\frac{\partial}{\partial r}u(r,\theta)\phi_p(r)\right]_a^b d\theta \\
&+& 2\pi \sum_{j=0}^{N_\theta} \hat{u}_{jq} \int_a^b r \sigma(r) \frac{d}{dr} \phi_j(r) \phi_p(r) dr \\
&+& 2\pi \sum_{j=0}^{N_\theta} \hat{u}_{jq} \int_a^b r^2 \sigma(r) \frac{d}{dr} \phi_j(r) \frac{d}{dr}\phi_p(r) dr \\
&+& 2\pi \sum_{j=0}^{N_\theta} \hat{u}_{jq} q^2 \int_a^b \sigma(r) \phi_j(r) \phi_p(r) dr.
\end{eqnarray}


Let us define the following matrices:
\begin{eqnarray}
({\bf M}_1)_{jp} & = & \int_a^b r \sigma(r) \frac{d}{dr} \phi_j(r) \phi_p(r) \;dr  \\
({\bf M}_2)_{jp} & = & \int_a^b r^2 \sigma(r) \frac{d}{dr} \phi_j(r) \frac{d}{dr} \phi_p(r) \;dr  \\
({\bf M}_3)_{jp} & = & \int_a^b \sigma(r) \phi_j(r) \phi_p(r) \; dr,
\end{eqnarray}
where $j, p = 0, \ldots, N_\theta$.

$T_4$ becomes:

\begin{eqnarray}
T_4 &=& \int_a^b \int_0^{2\pi} \phi_p(r) e^{iq\theta} r^2 f(r,\theta) d\theta dr \\
    &=& \int_a^b r^2\phi_p(r) \int_0^{2\pi} f(r,\theta) e^{iq\theta} d\theta dr \\
\end{eqnarray}

For given $r$, the Discrete Fourier Transform for $f(r, \theta)$ is defined by
\begin{equation}
f(r, \theta_\tau) = \frac{1}{2\pi} \sum_{k=-N_\theta/2+1}^{N_\theta/2} e^{ik\theta_\tau} \widehat{f(r)}_k
\end{equation}

where
\begin{equation}
\widehat{f(r)}_k = \frac{2\pi}{N_\theta} \sum_{j=1}^{N_\theta} e^{-ik\theta_j} f(r, \theta_j)
\end{equation}
with $ k \in \{-\frac{N_\theta}{2}+1, \cdots, \frac{N_\theta}{2}\}$ and $\theta_j \in \{ \frac{2\pi}{N_\theta}, \cdots, 2\pi  \}$.

Then:
\begin{center}
\begin{eqnarray}
T_4&=& \int_a^b r^2\phi_p(r) \int_0^{2\pi} \frac{1}{2\pi} \sum_{k=-N_\theta/2+1}^{N_\theta/2} e^{ik\theta} \widehat{f(r)}_k e^{iq\theta} d\theta dr \\
%&=& \int_a^b r^2\phi_p(r) \sum_{k=-N_\theta/2+1}^{N_\theta/2} \widehat{f(r)}_k \frac{1}{2\pi} \int_0^{2\pi}  e^{ik\theta} e^{iq\theta} d\theta dr \\
%&=& \int_a^b r^2\phi_p(r) \sum_{k=-N_\theta/2+1}^{N_\theta/2} \widehat{f(r)}_k \delta_{k,q} dr\\
&=& \int_a^b r^2\phi_p(r) \widehat{f(r)}_q dr.
%&=& \sum_\sigma w_\sigma r_\sigma^2\phi_p(r_\sigma) \widehat{f(r_\sigma)}_q.
\end{eqnarray}
\end{center}


\begin{eqnarray}
 2\pi \sum_{j=0}^{N_\theta} \hat{u}_{jq} {M_1}_{jp}
+ 2\pi \sum_{j=0}^{N_\theta} \hat{u}_{jq} {M_2}_{jp} + 2\pi
\sum_{j=0}^{N_\theta} \hat{u}_{jq} q^2 {M_3}_{jp} = \int_a^b
r^2\phi_p(r) \widehat{f(r)}_q dr \\
+ b^2\sigma(b) \phi_p(b) \int_0^{2\pi}e^{iq\theta} {\mathcal
G}_N(\theta) d\theta - a^2\sigma(a) \phi_p(a)
\int_0^{2\pi}e^{iq\theta} \frac{\partial}{\partial r}u(a,\theta)
d\theta
\end{eqnarray}
where $j, p = 0, \ldots, N_r$.

We apply the Fourier transform to the integral term with
${\mathcal G}_N$ and the same idea as one-dimensional case to each
term about boundary conditions.


\subsection{Experiment Results}
%\{ {\it  sp1d\_ch3\_1.tex} \}

\subsection {H/P Convergence Test for Two-dimensional Solution}

In this section we present the result of convergence in both $h$ refinement and $p$ refinement with the following steady-state Poisson differential equation:
\begin{equation}
\label{pois_sin}
-r^2 \frac{\partial}{\partial r} (\sigma \frac{\partial}{\partial r}u) - r \sigma \frac{\partial}{\partial r}u - \sigma \frac{\partial^2}{\partial \theta^2}u = r \cos \theta \left[ -4 r \pi \cos S_r + \{4\pi^2 + 1\} \sin S_r \right],
\end{equation}
for all $r \in [1, 2], \theta \in [0, 2\pi]$ with $\sigma(r) = r$. The analytic solution is known as 
\begin{equation}
u(r, \theta) = \sin S_r \cos \theta,
\end{equation}
where $S_r = 2\pi (r-1) - \pi$.

The numerical and exact solutions by the solver we developed is shown in figure (\ref{sinsol}).

\begin{figure}[h]
    \begin{center}
    \epsfig{file = figs_dn/sinDN_e10_t8.eps, width = 5cm} %Error: 4.7740e-15
    \epsfig{file = figs_dn/sinDN_e10_t8_1.eps, width = 5cm}
    \caption{\label{sinsol}Numerical and exact solution of equation (\ref{pois_sin}) with polynomial order $P=10$, $10$ equidistance elements. This gives the error 4.7740e-15 to the exact solution}
    \end{center}
\end{figure}

\begin{figure}[h]
    \begin{center}
    \epsfig{file = figs_dn/sinDNhconv.eps, width = 8.3cm}
    \epsfig{file = figs_dn/sinDNpconv.eps, width = 8.3cm}
    \caption{\label{sinDNconv}
(Left)Convergence with respect to discrete $L^{\infty}$ norm as a function of size of elements. This test is performed using the h-type extension with fixed polynomial order 5 and 6 respectively. Error on the Log-Log axis is demonstrating the algebraic convergence of the h-type extension.
(Right)Convergence w.r.t. $L^{\infty}$ norm as a function of size of polynomial order in semi-Log plot. It shows the exponential convergence of p-type extension for smooth solution. Two tests are performed for p-type extension with element length $0.2$ and $0.1$.
}
\end{center}
\end{figure}


\begin{table}[h]
\centering \caption{\label{hconv2t} This table shows the convergence of h-type (left) and p-type (right) resolution control done above Figure (\ref{sinDNconv}). We can see the slopes of each order $P$ is $P+1$ }
\begin{tabular}{|c|c|c|} \hline
    Polynomial order&Error($L^{\infty}$)&Slope   \\ \hline \hline
    5&$9.3603e-013$ &$5.9406$ \\ \hline
    6&$8.6542e-014$ &$6.9402$ \\ \hline
\end{tabular}
\hspace{.5in}
\begin{tabular}{|c|c|} \hline
    Element Size&Error($L^{\infty}$)\\ \hline
    0.2&$1.9385e-012$  \\ \hline
    0.1&$3.4611e-013$  \\ \hline
\end{tabular}

\end{table}

%\{ {\it  sp1d\_ch3\_2.tex} \}

\subsection {High order Polynomial Solution and Its Convergence}

In this section we construct a polynomial $P_n$ of order $n$ defined on $[0,1]$, which satisfies the following.
\begin{eqnarray}
\label{pois_scrv}
    P_n(0) = 0, &P_n(1) = 1 \\
    \frac{d^k}{dx^k}P_n(0) = 0, &\frac{d^k}{dx^k}P_n(1) = 0
\end{eqnarray}
for all $k = 1, \cdots, n-2$. \\
Then for each $n$, we obtain a polynomial $P_n$ by solving a system of linear equations having unique solution which determines the set of coefficients of $P_n$. We will apply the spectral polynomial solver to approximate the second derivative $Q_{n-2}$ of $P_n$.

\begin{figure}[h]
    \begin{center}
    \epsfig{file = figs_dn/ScrvH5_A.eps, width = 5cm}
    \epsfig{file = figs_dn/ScrvH5_N.eps, width = 5cm}
    \caption{\label{scrvsol}Example of curve that satisfies conditions (\ref{pois_scrv}) with polynomial order $n=7, 9, 12, 14, 15$. The error is $2.9239e-8$}
    \end{center}
\end{figure}

We will investigate the convergences by the h-type, p-type extension of trial functions below.

\begin{problem}
Consider the following differential equation for $u(x)$ such that
\begin{equation}
\label{poi_poly}
-r^2 \frac{\partial}{\partial r} (\sigma \frac{\partial}{\partial r}u) - r \sigma \frac{\partial}{\partial r}u - \sigma \frac{\partial^2}{\partial \theta^2}u = e^{-(r-1)^2}\{P_n(r) + (2r^2(r-1)-r)P_n'(r) - r^2 P_n''(r)\} \cos \theta,
\end{equation}
for all $r$ in $[0.1, 1.1]$ with $\sigma(r) = e^{-(r-1)^2}$. Then we have the exact solution as follows.
\begin{equation}
u(r, \theta) = P_n(r) \cos \theta
\end{equation}
where $r$ in $[0.1, 1.1]$ and $\theta$ in $[0, 2\pi]$.
\end{problem}

Note that the accuracy of this interpolation satisfying equations (\ref{pois_scrv}) is dependent on the stability of matrix that defines the coefficients of interpolants. We used the Legendre basis function which is known to be more stable than monomials. In spite of this, there exists interpolation error close to $e^{-13}$. This cause the same amount of convergence error in p-type extension mode shown in right of Figure (\ref{ScrvconvDN}) and Table (\ref{hconv2t}).

\begin{figure}[h]
\begin{center}
\epsfig{file = figs_dn/ScrvHconv.eps, width = 8.3cm}
\epsfig{file = figs_dn/ScrvPconv.eps, width = 8.3cm}
\caption{\label{ScrvconvDN}
(Left)Convergence with respect to discrete $L^{\infty}$ norm as a function of size of elements. This test is performed using the h-type extension with fixed polynomial order 3, 4, and 5 respectively. Error on the Log-Log axis is demonstrating the algebraic convergence of the h-type extension.
(Right)Convergence w.r.t. $L^{\infty}$ norm as a function of size of polynomial order in semi-Log plot. It shows the exponential convergence of p-type extension for smooth solution. Two tests are performed for p-type extension with element length $0.2$ and $0.1$.
}
\end{center}
\end{figure}

\begin{table}[h]
\centering \caption{\label{hconv2t} This table shows the convergence of h-type resolution control done above Figure (\ref{ScrvconvDN}). We can see the slopes of each order $P$ is $P+1$ }
\begin{tabular}{|c|c|c|} \hline
    Polynomial order&Error($L^{\infty}$)&Slope   \\ \hline \hline
    5&$8.5305e-012$ &$5.7556$ \\ \hline
    6&$4.7180e-012$ &$6.8332$ \\ \hline
\end{tabular}
\hspace{.5in}
\begin{tabular}{|c|c|} \hline
    {Element Size}&Error($L^{\infty}$) \\ \hline \hline
    0.2&$9.0616e-13$  \\ \hline
    0.1&$9.4747e-13$ \\ \hline
\end{tabular}
\end{table}


\clearpage
\section{Conclusion}
%\clearpage
%\{ {\it  sp1d\_ch4\_1.tex} \}

Throughout this project, we investigate the application of
Spectral Polynomial Element Method to Poisson equations. We also
compared the  of h/p convergence properties of the method to the
classical finite element method. The Galerkin method allows
incorporation of the weak solution into the formulation for the
problem as system of linear equations which can be solved
numerically.

The Spectral element solver for one dimensional Poisson equation
with Dirichlet and Neumann boundary conditions, with high-order
solutions exhibited much accurate solutions which were hard to get
acceptable convergence in a given time and resolution of domain in
past.

For the future study, we would like deal with problems regarding:

\begin{itemize}
\item
development multi-dimensional solver.
\item
obtaining solutions to various natural phenomena that obey
governing equations.
\item
utilizing the method in the ill-posed problem. By using certain
technique that we can approximate the solution, we can also apply
this method to the problem and compare with other method in that
situation.
\end{itemize}


%\end{spacing}
%------------------------------------------------------------------------------
%   BIBLIOGRAPHY
\input{ces_chbib}


%-------------------------------------------------------------------------------
%   APPENDIX


\end{document}
