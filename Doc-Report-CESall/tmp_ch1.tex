
%\{ {\it  sp1d\_ch1.tex} \}

Spectral method is a numerical scheme to approximate and simulate
the solution of partial differential equations. It has developed
rapidly in the past three decades and been applied to many field
in numerical simulation.

One of main reasons that it has gained broad and fast acceptance
is that it can take various system of infinitely differentiable
basis functions as trial functions, for instance, we can use the
representation of a function $u$ throughtout the domain via a
truncated serier exansion as follows:
\begin{equation}
\label{solapprx}
u \approx u_N = \sum_{n=0}^{N} \hat u_n \phi_n,
\end{equation}
where $\phi_n$ are the basis functions. In general this basis
functions may be the Chebyshev polynomials $T_n$ or the Legendre
polynomials $L_n$ or another member of the class of Jacobi
polynomials $P_n^{\alpha, \beta}$. By choosing an appropriate
orthogonal system based on its domain of orthogonality, we can
apply the method to problems such as periodic/non-periodic
problems and problems defined on compact domain, half/all
intervals.


Spectral methods can be classified into 2 parts in its
characteristics. We call them to be nodal and modal method,
respectively.
\begin{itemize}
\item The nodal method is usually called as pseudo-spectral or
collocation methods. The coefficients $\hat u_n$ of
(\ref{solapprx}) are obtained by requiring  the residual function
to be zero exactly at grid that is a set of nodes.

\item The modal method is associated with the method of weighted
residuals where the residual function is weighted with a set of
test functions and after integration is set to zero. In our
formulation we bring in the Galerkin method which has test
functions same as the basis functions.
\end{itemize}

In the collocation approach the coefficients represent the nodal
value of the physical variable unlike the Galerkin method.

The other thing which is so fascinating is its high accuracy and
convergence. In particular, the spectral polynomial method
facilitates the control of the resolution of element size and the
order of approximation. This enables the method to converge in
exponential speed which shows a noticeable difference from
classical finite difference and other purely element methods.
Unlike finite elements and finite difference, the order of
convergence is not fixed and it is related to the maximum
regularity of the solution.

In this report, I investigate the spectral polynomial method and
Fourier method for solving a partial differential equation
specific to the forward Poisson problem with Dirichlet and Neumann
boundary condition on inner and outer boundary circles
respectively. Throughout the report, I will  formulate our
approach that utilizes spectral polynomial element and Fourier
method in tensor product form. As an conclusion we present the
result of numerical solution and its convergence by h/p adaptive
control.
