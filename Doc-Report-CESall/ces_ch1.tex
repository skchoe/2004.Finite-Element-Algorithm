
%\{ {\it  sp1d\_ch1.tex} \}

The spectral method is a numerical scheme for approximating the
solution of partial differential equations. It has developed
rapidly in the past three decades and has been applied to
numerical simulation problems in many fields.

One reason for its broad and fast acceptance is its use of various
systems of infinitely differentiable basis functions as trial
functions. By choosing an appropriate orthogonal system based on
domain of orthogonality, we can apply the method to periodic and
non-periodic problems, as well as problems defined on various
domains, such as compact domain, half/all intervals.

Another advantage of the spectral method is its high accuracy. In
particular, the spectral polynomial method allows control of both
the resolution of element size and the order of approximation.
This results in exponential convergence, which is a marked
improvement over classical finite difference and other purely
element methods.

For solving the problem, we discretize the domain and obtain the
following approximation of infinite series. A function $u$ is
represented via a truncated series expansion as follows:
\begin{equation}
\label{solapprx}
u \approx u_N = \sum_{n=0}^{N} \hat u_n \phi_n,
\end{equation}
where $\phi_n$ are the basis functions. In general the Chebyshev
polynomials $T_n$ or the Legendre polynomials $L_n$ or another
member of the class of Jacobi polynomials $P_n^{\alpha, \beta}$
are employed as basis functions.

In choosing proper basis function, we apply the following 3
characteristics to the candidate basis.
\begin{itemize}
\item {\bf Numerical Efficiency} After discretization,
the choice of basis affects complexity of mass matrix. Moreover we
need consider the efficiency in solving the system of linear
equations. For example, using the monomials $\{x^k\}_{k=0}^{N}$
result in the matrix whose non-zero component are the half. But
its inverse matrix is full and the cost of inverting is dominant.
In contrast, the Legendre polynomial basis composes diagonal mass
matrix and its inverse matrix can be calculated very efficiently.
\item {\bf Conditioning} If a matrix system is ill-conditioned
the round-off error in the matrix system can lead to large errors
in the solution. Furthermore the number iteration required in
inverting the matrix using iterative solver can be increase by the
condition number. The condition number of monomials and Lagrange
polynomial are close to $10^p$, for the polynomial order $p$. But
Legendre polynomial has condition number $2P+1$. This condition
also affects the degree of linear independence of each basis
function.
\end{itemize}

This project seeks to study the fundamental theory of spectral
method, problems, solvability, and to obtain its constructive
procedure to apply the method to various application problems. By
checking solutions obtained via the spectral method against exact
solutions, we can validate the method and see how much we can save
the effort on discretisation of domain to achieve the same degree
of accuracy in comparison to classical element methods.

In chapter 2, I investigate spectral method solutions for the
forward Poisson problem with Dirichlet and Neumann boundary
condition on each end of one dimensional interval domain. In
chapter 3, I investigate the spectral polynomial method and
Fourier method for solving the forward Poisson problem with
Dirichlet and Neumann boundary condition on inner and outer
boundary circles respectively. Our approach utilizes spectral
polynomial element and Fourier method in tensor product form. As
an conclusion we present the result of numerical solution and its
convergence by h/p adaptive control.

\subsection{Poisson Equation}

Science and engineering disciplines are generally interested in
systems of continuous quantities and relations. This project
focuses on solutions of the Poisson equation, which appears in
various field such as electrostatics, magnetics, heat flow,
elastic membranes, torsion, and fluid flow.

For example, electrostatics, the governing equation appears as
Gauss's law in differential form \cite{Johnson}:
\begin{equation}
\nabla \cdot \mathbf{E} = 4 \pi \rho
\end{equation}
which indicates that the charge within a closed spherical surface
is related to the electric field \textbf{E} normal to surface
element where $\rho$ is a charge density.

Since it is known in electrostatics that the electric field \textbf{E} is conservative, \textbf{E} is a form of gradient of a scalar potential $\Phi$,
\begin{equation}
\mathbf{E} = - \nabla\Phi.
\end{equation}
With these two relationships, we obtain a Poisson equation
\begin{equation}
\nabla^2\Phi = -4\pi\rho.
\end{equation}


In this report, the one-dimensional Poisson equation is defined as

\begin{equation}
\label{poisson1} L(u) \equiv  \frac{d^2}{dx^2} u + f = 0.
\end{equation}
where $u$ and $f$ are defined on $\Omega$.

In pointwise viewpoint, the one dimensional Poisson equation
(\ref{poisson1}) is written as
\begin{equation}
\label{poisson2} L(u)(x) \equiv \frac{d^2}{dx^2} u(x) + f(x) = 0,
\end{equation}
for all $x$ in $(a, b)$.

\subsection{Method of Weighted Residuals}
According to the Weierstrass approximation theorem, for any given
real valued continuous solution $u$ on a compact interval $[a, b]$
we can obtain real polynomial function $p$ of certain degree such
that $p$ uniformly approximates $u$. Although the result of
convergence at each point is within a predefined error bound, this
does not satisfy the requirement that we need to acquire an
accurate solution on a specific situation. By imposing certain
restrictions, we can obtain a formulation that satisfies the
requirement.

To describe this, we set a general linear differential equation on $\Omega$.
\begin{equation}
\label{pde1} L(u) = 0.
\end{equation}
with appropriate initial and boundary conditions. Under certain
restriction,  we assume that the true solution $u(x)$ can be
approximated by a finite series expansion of the form:
\begin{equation}
\label{sol1} u^{\delta}(x) = \sum_{i=0}^{N_{dof}-1} \hat u_i
\Phi_i(x),
\end{equation}
where $\Phi_i(x)$ are polynomial trial functions and $\hat u_i$
are $N_{dof}$ unknown coefficients. We assume the following:
\begin{eqnarray}
    \hat u_0  &=& \mathcal{G}_{D} \qquad \mbox{: Dirichlet Boundary Value}, \\
    \Phi_0(a) &=& 1,  \Phi_{N_{dof}-1}(b) = 1 \qquad \mbox{where $a, b$ are the boundary of domain $\Omega$}\\
\end{eqnarray}
We then define a non-zero residual $R$ by:
\begin{equation}
R(u^{\delta}) = L(u^{\delta}).
\end{equation}

Define a set of functions, $H^1(\Omega)$  and a norm
$||\cdot||_{H^1(\Omega)}$ on the space as follows:
\begin{eqnarray}
H^1(\Omega) &=& \{v \in L^2(\Omega) : \frac{d}{dx}v \in L^2(\Omega)\}, \\
||v ||_{H^1(\Omega)} &=& \left[ \int_{\Omega}v(x)^2 + \frac{d}{dx}v(x)^2 dx \right]^{\frac{1}{2}}, \quad v \in H^1(\Omega).
\end{eqnarray}

We define an inner-product $\langle \cdot, \cdot \rangle$ over
$H^1(\Omega)$  as follows:
\begin{equation}
\label{functional}
\langle u, v \rangle = \int_{\Omega} u(x) \cdot v(x) dx,
\end{equation}

This method is restricted to test functions, $v(x)$ that satisfy:
\begin{equation}
\langle v, R \rangle = 0.
\end{equation}

For example, in the collocation method, the $j^{th}$ test function
is the Dirac delta function which evaluates to a collocation point
$x = x_j$. Then we have
\begin{equation}
0 = \langle \delta_j, R \rangle = \int_{\Omega} R(u^{\delta})(x)\delta_j(x)dx = R(u^{\delta})(x_j) = L(u^{\delta})(x_j).
\end{equation}

Other possible test functions are examined in \cite{Karniadarkis}.
